\documentclass[10pt,fleqn]{article} % Default font size and left-justified equations
\usepackage[%
    pdftitle={Modélisation systèmes multiphysiques : Modélisation linéaire et non linéaire},
    pdfauthor={Xavier Pessoles}]{hyperref}
    
\input{style/new_style}
\input{style/macros_SII}
\usepackage{multicol}
\usepackage{standalone}
\standaloneconfig{mode=buildnew}
\usepackage{siunitx}
\usepackage{wrapfig}
\fichetrue

%\fichefalse

%\proftrue
%\proffalse

\tdtrue
%\tdfalse

\courstrue
\coursfalse

\def\discipline{Sciences \\Industrielles de \\ l'Ingénieur}
\def\xxtete{Sciences Industrielles de l'Ingénieur}

\def\classe{PSI$\star$}
\def\xxnumpartie{\textsf{Cycle 01}}
\def\xxpartie{Modéliser le comportement linéaire et non linéaire des systèmes multiphysiques}


\def\xxnumchapitre{Chapitre 1 \vspace{.2cm}}
\def\xxchapitre{\hspace{.12cm} Modélisation multiphysique}


\def\xxtitreexo{\noindent Robot palettiseur Kuka}
\def\xxsourceexo{\hspace{.2cm} \footnotesize{CCP MP 2010}}


\def\xxposongletx{2}
\def\xxposonglettext{1.45}
\def\xxposonglety{20}
%\def\xxonglet{Part. 1 -- Ch. 3}
\def\xxonglet{\textsf{Cycle 02}}

\def\xxactivite{Colle 1}
\def\xxauteur{\textsl{Xavier Pessoles}}

\def\xxcompetences{%
\textsl{%
\textbf{Savoirs et compétences :}\\
\textit{****}
%Les sources sont associées par un \emph{hacheur série}. La détermination des grandeurs électriques associées à ce montage permet de conclure vis à vis du cahier des charges.
%\noindent \textbf{Résoudre :} à partir des modèles retenus :
%\begin{itemize}[label=\ding{112},font=\color{ocre}] 
%\item choisir une méthode de résolution analytique, graphique, numérique;
%\item mettre en \oe{}uvre une méthode de résolution.
%\end{itemize}
%\begin{itemize}[label=\ding{112},font=\color{ocre}] 
%\item \textit{Rés -- C1.1 :} Loi entrée sortie géométrique et cinématique -- Fermeture géométrique.
%\end{itemize}
%
%\noindent \textit{Mod2 -- C4.1 :} Représentation par schéma-blocs.
}}

\def\xxfigures{
\includegraphics[width=.3\linewidth]{images/fig_01}
}%figues de la page de garde


\def\xxpied{%
Cycle 01 -- Modéliser le comportement des systèmes multiphysiques\\
\xxactivite%
}

\setcounter{secnumdepth}{5}
%---------------------------------------------------------------------------
%---------------------------------------------------------------------------


\begin{document}
%\chapterimage{png/Fond_Cin}
\input{style/new_pagegarde}
\vspace{4.5cm}
\pagestyle{fancy}
\thispagestyle{plain}


\def\columnseprulecolor{\color{ocre}}
\setlength{\columnseprule}{0.4pt} 

%\ifprof
%\else
\begin{multicols}{2}
%\fi
\section*{Mise en situation}
\ifprof
\else
\fi

Le robot Kuka, objet de cette étude, a pour objectif la palettisation de bidons utilisés en agriculture biologique (compléments permettant d'améliorer les qualités nutritives des produits agricoles).


\begin{obj}
On s’intéresse à l’asservissement en position de l’axe A1. On souhaite s’assurer que la chaîne 
fonctionnelle d’asservissement permet de  respecter  les performances  souhaitées en  terme de 
précision, rapidité et stabilité tout en restant peu sensible aux variations de l’inertie du robot 
suivant la charge transportée. 
\end{obj}

L’axe  $A_1$  est  mu  par  un  servomoteur  qui  présente  l'avantage  de  posséder  une  très  faible inertie. Le comportement électromécanique de ce type de moteur est donné par les équations suivantes : 
$u(t) = Ri(t) + e(t)$, $e(t) =k_e \omega_m(t)$, $J_e \dfrac{\dd \omega_mt(t)}{\dd t}=c_m(t)$, $c_m(t)=k_t i(t)$.

Avec  $u(t)$  la  tension  appliquée  aux  bornes  du moteur,  $i(t)$  le  courant  d’induit,  $e(t)$  la  force 
contre  électromotrice, $\omega_m(t)$  la  vitesse  de  rotation  du moteur,  $c_m(t)$  le  couple  délivré  par  le 
moteur et Je l’inertie équivalente ramenée sur l’arbre moteur. 
Le  réducteur  retenu  pour  cette  motorisation  est  un  réducteur  Harmonic-Drive.  Les 
caractéristiques de l’ensemble moteur-réducteur sont les suivantes : 
\begin{itemize}
\item $k_e = \SI{0,2}{V/(rad/s)}$ : constante de force électromotrice ; 
\item $k_t = \SI{0,2}{Nm/A}$ : constante de couple ; 
\item $R = \SI{2}{W}$ : résistance de l’induit ; 
\item $J_m = \SI{4e-3}{kg.m2}$ : inertie de l’ensemble axe moteur et réducteur sur l'arbre moteur ; 
\item $N = 200$ : rapport de transmission.
\end{itemize} 
L’inertie $J_1$ du robot autour de l’axe $\axe{O_1}{z_1}$ dépend de la configuration du robot et de la masse 
transportée. Elle est telle que : 
\begin{itemize}
\item $J_{1,\text{mini}} = \SI{50}{kg.m^2}$ lorsque le déplacement a lieu à vide ; 
\item $J_{1,\text{maxi}} = \SI{200}{kg.m^2}$ lorsque la masse transportée est de \SI{50}{daN}.
\end{itemize}

L’inertie équivalente Je ramenée sur l’arbre moteur est alors égale à : 
\begin{itemize}
\item $J_{e,\text{mini}} = \SI{5,25e-3}{kg.m^2}$ lorsque $J_1=J_{1,\text{mini}}$ ; 
\item $J_{e,\text{maxi}} =\SI{9e-3}{kg.m^2}$ lorsque $J_1=J_{1,\text{maxi}}$.
\end{itemize}

La chaîne fonctionnelle de l’asservissement de l’axe $A1$ est représentée ci-dessous. La boucle interne réalise une correction de vitesse à partir de la tension $u_g(t)$ fournie par une 
génératrice  tachymétrique  de  gain $K_g$ montée  en  prise  directe  sur  le moteur. $G$  est  le  gain 
réglable de l’amplificateur de puissance. 

\begin{center}
\includegraphics[width=\linewidth]{images/fig_02}

\end{center}
La boucle externe  réalise  la  correction de position à partir de  la  tension $u_r(t)$  fournie par  le 
capteur de position de gain $K_r$ monté  en prise directe  sur  l’arbre de  sortie du  réducteur. La 
fonction de transfert du correcteur est notée $C(p)$. 



\end{multicols}
\end{document}

\subparagraph{}\textit{}
\ifprof
\begin{corrige}~\\
\end{corrige}
\else
\fi

\begin{center}
\includegraphics[width=\linewidth]{images/img_04}
%\textit{}
\end{center}


\subparagraph{}\textit{}
\ifprof
\begin{corrige}~\\
\end{corrige}
\else
\fi

\subparagraph{}\textit{}
\ifprof
\begin{corrige}~\\
\end{corrige}
\else
\fi

\subparagraph{}\textit{}
\ifprof
\begin{corrige}~\\
\end{corrige}
\else
\fi

\subparagraph{}\textit{}
\ifprof
\begin{corrige}~\\
\end{corrige}
\else
\fi

\subparagraph{}\textit{}
\ifprof
\begin{corrige}~\\
\end{corrige}
\else
\fi

\subparagraph{}\textit{}
\ifprof
\begin{corrige}~\\
\end{corrige}
\else
\fi
\begin{center}
%\includegraphics[width=\linewidth]{images/fig_05}
\end{center}
