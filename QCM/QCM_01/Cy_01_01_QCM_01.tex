\documentclass[10pt,fleqn]{article} % Default font size and left-justified equations
\usepackage[%
    pdftitle={Modélisation systèmes multiphysiques : Modélisation linéaire et non linéaire},
    pdfauthor={Xavier Pessoles}]{hyperref}
    
\input{style/new_style}
\input{style/macros_SII}
\usepackage{multicol}
\usepackage{siunitx}
\fichetrue
%\fichefalse

\proftrue
\proffalse

\tdtrue
%\tdfalse

\courstrue
\coursfalse

\def\discipline{Sciences \\Industrielles de \\ l'Ingénieur}
\def\xxtete{Sciences Industrielles de l'Ingénieur}

\def\classe{PSI$\star$ -- MP}
\def\xxnumpartie{\textsf{Cycle 01}}
\def\xxpartie{Modéliser le comportement linéaire et non linéaire des systèmes multiphysiques}


\def\xxnumchapitre{Chapitre 1 \vspace{.2cm}}
\def\xxchapitre{\hspace{.12cm} Modélisation multiphysique}


\def\xxtitreexo{QCM}%Motorisation du moteur Haibike}
\def\xxsourceexo{\hspace{.2cm} \footnotesize{P. Beynet, Éd Ellipses.}}


\def\xxposongletx{2}
\def\xxposonglettext{1.45}
\def\xxposonglety{20}
%\def\xxonglet{Part. 1 -- Ch. 3}
\def\xxonglet{\textsf{Cycle 01}}

\def\xxactivite{QCM 01}
\def\xxauteur{\textsl{P. Beynet, Éd Ellipses.}}

\def\xxcompetences{%
\textsl{%
\textbf{Savoirs et compétences :}\\
%Les sources sont associées par un \emph{hacheur série}. La détermination des grandeurs électriques associées à ce montage permet de conclure vis à vis du cahier des charges.
%\noindent \textbf{Résoudre :} à partir des modèles retenus :
%\begin{itemize}[label=\ding{112},font=\color{ocre}] 
%\item choisir une méthode de résolution analytique, graphique, numérique;
%\item mettre en \oe{}uvre une méthode de résolution.
%\end{itemize}
%\begin{itemize}[label=\ding{112},font=\color{ocre}] 
%\item \textit{Rés -- C1.1 :} Loi entrée sortie géométrique et cinématique -- Fermeture géométrique.
%\end{itemize}
%
%\noindent \textit{Mod2 -- C4.1 :} Représentation par schéma bloc.
}}

\def\xxfigures{
%\includegraphics[width=.9\linewidth]{images/c-evolution}
}%figues de la page de garde


\def\xxpied{%
Cycle 01 -- Modéliser le comportement des systèmes multiphysiques\\
Chapitre 1 -- \xxactivite%
}

\setcounter{secnumdepth}{5}
%---------------------------------------------------------------------------

\usepackage{pgfplots}
\begin{document}
\def\pathfig{images}
%\chapterimage{png/Fond_Cin}
\input{style/new_pagegarde}
\vspace{8cm}
\pagestyle{fancy}
\thispagestyle{plain}

\def\columnseprulecolor{\color{ocre}}
\setlength{\columnseprule}{0.4pt} 

\def\pathfig{images}

%\begin{multicols}{2}
%\end{multicols}

Vrai ou faux : 
\begin{enumerate}
\item La modélisation d’un système dans le domaine symbolique ne concerne que les systèmes linéaires. 
\item Dans un schéma-blocs, les liens sont des grandeurs physiques dont les conditions initiales sont nulles. 
\item La modélisation par schéma-blocs d’un système est unique.
\item La modélisation par schéma-blocs dans le domaine symbolique impose des règles d’association des différents blocs. On ne peut donc relier n’importe quel bloc à un autre.
\item Les modélisations causale et acausale d’un même système, donnent des résultats simulés identiques.
\item La modélisation acausale nécessite de la part du modélisateur une connaissance des lois de comportement des constituants.
\item Il est possible de mélanger modélisation causale et modélisation acausale dans un modèle global.
\item Cette modélisation respecte les règles d’association : 
\item Cette modélisation respecte les règles d’association : 
\item Dans la formule donnant la puissance électrique instantanée : $p(t)=u(t)\times i(t)$, $u(t)$ est une grandeur de type <<~effort~>> et est une grandeur de type <<~flux~>>.

\end{enumerate}

\ifprof
\begin{corrige}
\begin{enumerate}
\item Vrai : en effet, la modélisation dans le domaine symbolique ne concerne que les systèmes linéaires.
\item Vrai : la modélisation sous forme de fonction de transfert ne peut se faire que si les conditions initiales des fonctions du temps en entrée et en sortie sont nulles.
\item Faux : la modélisation par schéma-blocs est la traduction d’un système d’équations. On peut donner des schéma-blocs mathématiquement équivalent, c’est-à-dire ayant le même comportement ou la même fonction de transfert globale.
\item Faux : dans la modélisation par schéma-blocs dans le domaine symbolique, les différents blocs peuvent théoriquement être reliés entre-eux, indépendamment de la nature des grandeurs physiques d’entrée et de sortie. Le modélisateur s’assurera tout de même que le modèle est cohérent : la grandeur de sortie d’un bloc modélisé doit être identique à la grandeur d’entrée du bloc auquel il est relié.
\item Vrai : dans le cas où le système est modélisable en modélisation causale et en modélisation acausale, les résultats sont bien identiques.
\item Faux : en modélisation acausale, le modélisateur ne doit renseigner que la valeur des paramètres influents. Sa connaissance des lois de comportement n’est pas nécessaire.
\item Vrai : on peut en effet mélanger modélisation causale et modélisation acausale à partir du moment où le modélisateur respecte la nature des flux véhiculés.
\item Vrai : les règles d’association sont respectées étant donné que les ports reliés sont de nature identique.
\item Faux : les règles d’association ne sont pas respectées étant donné que le port carré de la force imposée est relié à un lien joignant deux ports circulaires (flux d’énergie mécanique en rotation).
\item Vrai :en effet, dans la formule donnant la puissance électrique instantanée $p(t)=u(t) \times i(t)$, $u(t)$ est une grandeur de type « effort» et $i(t)$ est une grandeur de type <<~flux~>>.
\end{enumerate}
\end{corrige}
\else
\fi







\end{document}