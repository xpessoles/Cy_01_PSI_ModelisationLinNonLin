\documentclass[10pt,fleqn]{article} % Default font size an\item left-justifie\item equations
\usepackage[%
    pdftitle={Modélisation systèmes multiphysiques : Modélisation linéaire et non linéaire},
    pdfauthor={Xavier Pessoles}]{hyperref}
    
\input{style/new_style}
\input{style/macros_SII}
\usepackage{multicol}
\usepackage{siunitx}
\fichetrue
%\fichefalse

\proftrue
%\proffalse

\tdtrue
%\tdfalse

\courstrue
\coursfalse

\def\discipline{Sciences \\Industrielles de \\ l'Ingénieur}
\def\xxtete{Sciences Industrielles de l'Ingénieur}

\def\classe{PSI$\star$ -- MP}
\def\xxnumpartie{\textsf{Cycle 01}}
\def\xxpartie{Modéliser le comportement linéaire et non linéaire des systèmes multiphysiques}


\def\xxnumchapitre{Chapitre 1 \vspace{.2cm}}
\def\xxchapitre{\hspace{.12cm} Modélisation multiphysique}


\def\xxtitreexo{QCM}%Motorisation du moteur Haibike}
\def\xxsourceexo{\hspace{.2cm} \footnotesize{X. Pessoles}}%P. Beynet, É\item Ellipses.}}


\def\xxposongletx{2}
\def\xxposonglettext{1.45}
\def\xxposonglety{20}
%\def\xxonglet{Part. 1 -- Ch. 3}
\def\xxonglet{\textsf{Cycle 01}}

\def\xxactivite{QCM 02}
\def\xxauteur{\textsl{X. Pessoles}}

\def\xxcompetences{%
\textsl{%
\textbf{Savoirs et compétences :}\\
%Les sources sont associées par un \emph{hacheur série}. La détermination des grandeurs électriques associées à ce montage permet de conclure vis à vis du cahier des charges.
%\noindent \textbf{Résoudre :} à partir des modèles retenus :
%\begin{itemize}[label=\ding{112},font=\color{ocre}] 
%\item choisir une méthode de résolution analytique, graphique, numérique;
%\item mettre en \oe{}uvre une méthode de résolution.
%\end{itemize}
%\begin{itemize}[label=\ding{112},font=\color{ocre}] 
%\item \textit{Rés -- C1.1 :} Loi entrée sortie géométrique et cinématique -- Fermeture géométrique.
%\end{itemize}
%
%\noindent \textit{Mod2 -- C4.1 :} Représentation par schéma bloc.
}}

\def\xxfigures{
%\includegraphics[width=.9\linewidth]{images/c-evolution}
}%figues de la page de garde


\def\xxpied{%
Cycle 01 -- Modéliser le comportement des systèmes multiphysiques\\
Chapitre 1 -- \xxactivite%
}

\setcounter{secnumdepth}{5}
%---------------------------------------------------------------------------

%\usepackage{pgfplots}
\begin{document}
\def\pathfig{images}
%\chapterimage{png/Fond_Cin}
\input{style/new_pagegarde}
\vspace{6cm}
\pagestyle{fancy}
\thispagestyle{plain}

\def\columnseprulecolor{\color{ocre}}
\setlength{\columnseprule}{0.4pt} 

\def\pathfig{images}

\begin{multicols}{2}
%\end{multicols}



\subparagraph{}\textit{Cocher les propositions justes}

\begin{center}
\includegraphics[width=.65\linewidth]{images/fig_01}
\end{center}


\begin{enumerate}
\item Il s'agit de la réponse indicielle d'un système du premier ordre.
\item Il s'agit de la réponse indicielle d'un système du deuxième ordre.
\item Le coefficient d'amortissement est plus gran\item que 1.
\item Le coefficient d'amortissement est plus petit que 0,7.
\item Le coefficient d'amortissement est plus petit que 1.
\end{enumerate}

\subparagraph{}\textit{Cocher les propositions justes.}
\begin{center}
\includegraphics[width=.65\linewidth]{images/fig_01}
\end{center}

\begin{enumerate}
\item Je suis sensé savoir calculer le coefficient d'amortissement à partir du
premier dépassement.
\item Je suis sensé savoir calculer le coefficient d'amortissement à partir du
secon dépassement.
\item Je suis sensé savoir calculer le coefficient d'amortissement à partir de la pseudo période.
\item Je suis sensé savoir calculer le coefficient d'amortissement à partir du temps de pic.
E Je suis sensé savoir calculer le coefficient d'amortissement à partir de la valeur finale.
F Je suis sensé savoir calculer la pulsation à partir du temps de réponse à 5%.
\end{enumerate}


\subparagraph{}\textit{Cocher les propositions justes.}

\begin{center}
\includegraphics[width=.65\linewidth]{images/fig_01}
\end{center}

\begin{enumerate}
\item Je suis sensé savoir calculer la pulsation propre à partir du premier
dépassement.
\item Je suis sensé savoir calculer la pulsation propre à partir du second
dépassement.
\item Je suis sensé savoir calculer la pulsation propre à partir de la pseudo
période.
\item Je suis sensé savoir calculer la pulsation propre à partir du temps de pic.
\item Je suis sensé savoir calculer la pulsation propre à partir de la valeur finale.
\item Je suis sensé savoir calculer la pulsation propre à partir du temps de réponse à 5%.
\end{enumerate}



\subparagraph{}\textit{Cocher les propositions justes.}

\begin{center}
\includegraphics[width=.65\linewidth]{images/fig_01}
\end{center}
 
\begin{enumerate}
\item Le premier dépassement est égal à 1,5.
\item Le premier dépassement est égal à 0,5.
\item Il n'y a pas de dépassement.
\item La réponse D.
\end{enumerate}



\subparagraph{}\textit{T désigne la constante de temps d'un système du premier ordre.}
\begin{enumerate}
\item Le temps de réponse à 5% est de 3T.
\item Le signal atteint 63% de la valeur finale à T.
\item Le temps de réponse à 5% est de 5T.
\item Je ne sais pas.
\end{enumerate}



\subparagraph{}\textit{Le gain statique d'un système d'ordre 1 est K. Le système est sollicité par un échelon d'amplitude E.}
\begin{enumerate}
\item La valeur finale vaut K.
\item La valeur finale vaut E.
\item La valeur finale vaut KE.
\item La valeur finale vaut 0.
\item La valeur finale vaut 12.
\item La valeur finale est infinie.
\end{enumerate}


\subparagraph{}\textit{ Le gain statique d'un système d'ordre 1 est K. Le système est sollicité par une rampe de pente a.}
\begin{enumerate}
\item La valeur finale vaut K.
\item La valeur finale vaut E.
\item La valeur finale vaut KE.
\item La pente de l'asymptote est Ka.
\item L'écart statique est non nul.
\item La valeur finale est infinie.
\end{enumerate}


\subparagraph{}\textit{Parmi les propositions suivantes, lesquelles sont vraies ?}
\begin{enumerate}
\item Pour un coefficient d'amortissement de 0,7, le temps de réponse est le plus rapide avec
dépassement.
\item Pour un coefficient d'amortissement de 1, le temps de réponse est le plus rapide ave\item dépassement.
\item Pour un coefficient d'amortissement de 0,7, le temps de réponse est le plus rapide sans
dépassement.
\item Pour un coefficient d'amortissement de 1, le temps de réponse est le plus rapide sans dépassement.
\item Le système est plus rapide si le coefficient d'amortissement vaut 0,7 que s'il vaut 1.
\item Le système est plus rapide si le coefficient d'amortissement vaut 0,2 que s'il vaut 1.
\end{enumerate}

\subparagraph{}\textit{Les (La) caractéristique(s) du diagramme de Bode d'un intégrateur K/p sont (est) :}
\begin{enumerate}
\item le gain a une pente de - 20dB/decade.
\item le gain a une pente de + 20 dB/decade.
\item le gain a une pente nulle.
\item le gain passe par le point (1,20 logK)
\item le gain passe par le point (0,20 logK)
\item la phase est de +90°.
\item la phase est de -90°.
\end{enumerate}

\subparagraph{}\textit{Les (La) caractéristique(s) du diagramme de Bode d'un dérivateur Kp sont (est)
:}
\begin{enumerate}
\item le gain a une pente de - 20dB/decade.
\item le gain a une pente de + 20 dB/decade.
\item le gain a une pente nulle.
\item le gain passe par le point (1,20 logK)
\item le gain passe par le point (0,20 logK)
\item la phase est de +90°.
\item la phase est de -90°.
\end{enumerate}

\subparagraph{}\textit{Soit un système du premier ordre de gain K et de constante de temps T. On
souhaite tracer le diagramme de Bode de la fonction de transfert.}
\begin{enumerate}
\item Le gain est forcément négatif.
\item Le gain est forcément positif.
\item Le gain est forcément positif puis négatif.
\item Le gain peut être positif et négatif.
\item Le gain peut être négatif.
\item La rupture de pente est à 1/T.
\item La rupture de pente est à T.
\item Il n'y a pas de rupture de pente.
\end{enumerate}

\subparagraph{}\textit{Soit un système d'ordre 2 de gain K, d'amortissement z, de pulsation propre
w0. On souhaite tracer e diagramme de Bode.}
\begin{enumerate}
\item Si z > 1 il y a deux ruptures de pente.
\item Si z > 1 il y a une rupture de pente.
\item Lorsque w ten\item vers l'infini, la pente de l'asymptote du gain est de -40 dB/decade.
\item Lorsque w ten\item vers 1, la pente du gain est de 0.
\end{enumerate}

\subparagraph{}\textit{Soit un système d'ordre 2 de gain K, d'amortissement z<0.7, de pulsation
propre w0. On souhaite tracer le diagramme de Bode.}
\begin{enumerate}
\item La pulsation de résonance est inférieure à la pulsation propre.
\item On peut déterminer le coefficient d'amortissement à partir du gain à la résonance.
\item On peut déterminer le gain à partir de la valeur du gain lorsque w ten\item vers 0.
\item On peut déterminer la pulsation propre à partir de l'intersection des asymptotes.
\end{enumerate}



\subparagraph{}\textit{L'exosquelette qui faisait l'objet de l'exercice du jour était :}
\begin{enumerate}
\item est un système suiveur
\item est un système régulateur
\item n'est pas vraiment un système asservi
\item la réponse D
\item je suis désolé Monsieur, j'ai pas fait l'exo, j'ai fait un exo de physique.
\end{enumerate}

\end{multicols}
\end{document}