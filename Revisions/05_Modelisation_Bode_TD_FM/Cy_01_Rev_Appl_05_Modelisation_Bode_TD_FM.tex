\documentclass[10pt,fleqn]{article} % Default font size and left-justified equations
\usepackage[%
    pdftitle={Modélisation systèmes multiphysiques : Analyse fréquentielle},
    pdfauthor={Xavier Pessoles}]{hyperref}

\input{style/new_style}
\input{style/macros_SII}

\fichetrue
%\fichefalse

\proftrue
\proffalse

\tdtrue
%\tdfalse

%\courstrue
\coursfalse

%%%%%%%%%%%%%%%%%%%
\usepackage{numprint}
\usetikzlibrary{calc}
%%definition style pour mettre un fond blanc dans un node sans avoir des marges énormes
%\tikzset{fondblanc/.style={ inner sep=2pt,fill=white,outer sep = 5pt}} 
%\tikzset{fondblanc2/.style={ inner sep=2pt,fill=white}} 
%\tikzset{fondblanc3/.style={ inner sep=1pt,fill=white,outer sep = 2pt}} 
%\usetikzlibrary{calc,circuits.ee.IEC}
%\usetikzlibrary{shapes}
%\usepackage[european resistor, european voltage, european current]{circuitikz}
%\usetikzlibrary{babel}
%\usepackage{standalone}
%\standaloneconfig{mode=buildnew}
%\usepackage{style/schemabloc}
%%\usepackage{picins}
\usepackage{siunitx}
% -------------------------------------
% Déclaration des titres
% -------------------------------------

\def\discipline{Sciences \\Industrielles de \\ l'Ingénieur}
\def\xxtete{Sciences Industrielles de l'Ingénieur}


\def\classe{Cy01 - R 05}
\def\xxnumpartie{Cycle 01}
\def\xxpartie{Modéliser le comportement linéaire et non linéaire des systèmes multiphysiques}

\def\xxnumchapitre{Révisions 5 \vspace{.2cm}}
\def\xxchapitre{\hspace{.12cm} Modélisation des systèmes linéaires -- Domaine fréquentiel}

\def\xxposongletx{2}
\def\xxposonglettext{1.45}
\def\xxposonglety{10}%19

\def\xxonglet{Cycle 01 -- Rév 5}

\def\xxactivite{TD}
\def\xxtitreexo{Étude dynamique d’un train d’atterrissage}
\def\xxsourceexo{D'après ressources de Florestan Mathurin \url{http://florestan.mathurin.free.fr}}
\def\xxauteur{\textsl{Florestan Mathurin}}

\def\xxcompetences{%
\textsl{%
\textbf{Savoirs et compétences :}\\
}}

\def\xxfigures{
\includegraphics[width=.5\linewidth]{images/img_01}%images/prot_01}
}%figues de la page de garde

\def\xxpied{%
Cycle 01 -- Modéliser le comportement des systèmes multiphysiques\\
Révision 5 -- \xxactivite%
}

\setcounter{secnumdepth}{5}
%---------------------------------------------------------------------------


\begin{document}
%\chapterimage{images/Fond_Cin}
\input{style/new_pagegarde}
\vspace{5cm}
\pagestyle{fancy}
\thispagestyle{plain}

\def\columnseprulecolor{\color{ocre}}
\setlength{\columnseprule}{0.4pt} 


\ifprof
\else
\begin{multicols}{2}
\fi
\newpage

\section*{Présentation générale système}

L’amortisseur de train d’atterrissage d’avion est un système qui doit principalement amortir l’impact de l’avion à l’atterrissage mais qui doit aussi filtrer les sollicitations (pour les passagers, les équipements) imposées par les irrégularités du sol lorsque l’avion
est en phase de roulage sur la piste.

L’objectif du travail est de montrer que le respect du cahier des charges à l’impact lors du
dimensionnement de l’amortisseur est peu compatible avec les performances requises au
roulage.

\section*{Étude en phase d'impact}


On s’intéresse au comportement du train d’atterrissage lorsque l’avion est en phase d’atterrissage. On
considère que le train d’atterrissage dans cette phase de fonctionnement subit un impact modélisé par un
effort vertical $F_v(t)$.


\begin{center}
\includegraphics[width=\linewidth]{images/img_02}
\end{center}

La modélisation simplifiée de l'avion avec son
amortisseur est donnée sur la figure de gauche. À partir
de cette modélisation, on obtient la fonction de
transfert qui lie la hauteur $z_a(t)$ de l'avion à l’effort de
l’impact $F_v(t)$ au moment de l’impact :
$$
H_1(p)=\dfrac{Z_a(p)}{F_v(p)}=\dfrac{1}{K+cp+M_e p^2}
$$

Avec : 
\begin{itemize}
\item $Z_a(p)$ : transformée de Laplace de $z_a(t)$;
\item $F_v(p)$ : transformée de Laplace de $F_v(t)$;
\item $M_e$ : masse équivalente suspendue;
\item $K$ : raideur du ressort;
\item $c$ : coefficient d’amortissement visqueux;
\item $F_v(t)$ sera modélisé par une impulsion :
\item $F_v(t)=n M_e g \delta (t)$ avec $n$ facteur de charge.
\end{itemize}

\begin{center}
\includegraphics[width=\linewidth]{images/img_03}
\end{center}



\subparagraph{}
\textit{Montrer à l’aide du PFD que l’on peut obtenir la fonction de transfert $H_1(p)$. Donner l’ordre de la fonction
de transfert $H_1(p)$ puis l’expression du gain statique $G_0$ et de la pulsation propre $\omega_0$ en fonction de $K$ et $M_e$.
Donner ensuite $F_v(p)$, transformée de Laplace de $F_v(t)$.}
\ifprof
\begin{corrige}

\end{corrige}
\else
\fi



\subparagraph{}
\textit{Le corps humain peut tolérer au mieux des sollicitations verticales autour de \SI{1,6}{Hz}. On choisit donc $\omega_0 = 2\pi 1,6 \simeq \SI{10}{rad.s^{-1}}$. Calculer les valeurs numériques de $K$, $G_0$ et de $F_v(p)$ (on prendra $g \simeq \SI{10}{m.s^{-2}}$).}
\ifprof
\begin{corrige}
\end{corrige}
\else
\fi


\subparagraph{}
\textit{Modéliser sous Scilab-Xcos, le système ci-dessous ayant pour fonction de transfert :
$H_1(p)=\dfrac{Z_a(p)}{F_v(p)}=\dfrac{G_0}{1+ap+\dfrac{p^2}{\omega_0^2}}$ où $a =\dfrac{c}{k}$.}


\begin{center}
\includegraphics[width=\linewidth]{images/img_04}
\end{center}




\subparagraph{}
\textit{Rechercher, à l’aide de la simulation, la valeur de a la plus grande possible (à 0,005 prêt) afin de respecter
la course maximale donnée par le cahier des charges. En déduire le coefficient d’amortissement z
correspondant et le type de régime de la réponse indicielle (apériodique, critique ou pseudopériodique).}
\ifprof
\begin{corrige}
\end{corrige}
\else
\fi



\subparagraph{}
\textit{On préfère retenir par la suite une valeur de a permettant d’obtenir le régime critique en réponse indicielle. Déterminer cette nouvelle valeur de $a$. Donner à l’aide de la simulation la valeur de la course $z$ obtenue pour cette nouvelle valeur de $a$. Conclure vis-à-vis du cahier des charges.}
\ifprof
\begin{corrige}
\end{corrige}
\else
\fi


\subsection*{Étude en phase de filtrage des irrégularités de la piste}

\begin{center}
\includegraphics[width=\linewidth]{images/img_05}
\end{center}

La modélisation simplifiée de l'avion avec son
amortisseur dans cette phase de vie est
donnée sur la figure de gauche. À partir de
cette modélisation, on obtient la fonction de
transfert qui lie la hauteur $z_a(t)$ de l'avion à la
hauteur des aspérités du sol $z_p(t)$ :
$$
H_2(p)=\dfrac{Z_a(p)}{Z_p(p)}=\dfrac{K+cp}{K+cp+M_e p^2}.
$$

Avec :
\begin{itemize}
\item $Z_a(p)$ : transformée de Laplace de $z_a(t)$;
\item $Z_p(p)$ : transformée de Laplace de $z_p(t)$;
\item $M_e$ : masse équivalente suspendue;
\item $K$ : raideur du ressort;
\item $c$ : coefficient d’amortissement visqueux.
\end{itemize}



\subparagraph{}
\textit{Mettre la fonction de transfert $H_2(p)$ sous forme canonique et donner l’expression correspondante de ses paramètres (gain statique $k$, constante de temps $\tau$ du numérateur, coefficient d’amortissement $z$ et pulsation
propre non amortie $\omega_0$). Montrer que $H_2(p)$ peut se mettre sous la forme $\dfrac{Z_a(p)}{Z_p(p)}=\dfrac{1+0,2p}{\left(1+0,1p\right)^2}$.}
\ifprof
\begin{corrige}
\end{corrige}
\else
\fi

Modéliser sous Xcos le système ci-dessous en
utilisant une entrée sinusoïdale (amplitude : \SI{10}{mm}, pulsation : variable formelle wm,
déphasage : \SI{0}{rad}).

Réaliser une étude fréquentielle à l’aide d’un lieu de Bode dans le domaine fréquentiel allant de \SI{1}{rad/s} à \SI{1000}{rad/s}. On doit obtenir le diagramme document réponse 1.

\subparagraph{}
\textit{Tracer le diagramme asymptotique sur le document réponse 1. Tracer le point correspondant à $\omega=\SI{20}{rad/s}$ sur la courbe de phase.}
\ifprof
\begin{corrige}
\end{corrige}
\else
\fi


\subparagraph{}
\textit{Donner les expressions de $H_2(p)$ lorsque $p$ tend vers 0 puis lorsque $p$
tend vers $+\infty$. En déduire les systèmes simples que l’on peut utiliser afin de simplifier la réponse harmonique du système pour des pulsations très faibles et très importantes.}
\ifprof
\begin{corrige}
\end{corrige}
\else
\fi

On modélise les irrégularités de la piste par une fonction de la forme $z_p(t) = z_{p0}\sin(Fi_.y(t))$ où $z_{p0}$ est l’amplitude
de ces irrégularités (ici $z_{p0}=\SI{10}{mm}$, c’est l’amplitude que l’on a utilisée pour modéliser l’entrée dans le schéma-blocs dans la simulation) et où $F_i$ est une constante liée à la fréquence spatiale des irrégularités sur la piste 
(par exemple pour $F_i = \SI{1}{rad/m}$, on a une bosse tous les 6 mètres environ, ce qui
 correspond au changement de plaque de béton sur la piste). 
 On considère que l’avion est animé d’une vitesse horizontale 
 constante $\dot{y}_0$ telle que
 $\dot{y}_0 = \dfrac{\omega}{F_i}$  et 
 $y(t)=\omega t /F_i$.


\subparagraph{}
\textit{Déterminer à partir du diagramme de Bode, la relation liant $z_a(t)$ à $z_p(t)$ aux très faibles vitesses. Que se passe-t-il physiquement pour les passagers ?}
\ifprof
\begin{corrige}
\end{corrige}
\else
\fi


\subparagraph{}
\textit{Vérifier ces conclusions en lançant une simulation dans le domaine temporel en prenant par exemple $\dot{y}_0=\SI{2,6}{km/h}$ (horizon temporel = \SI{20}{s}).}
\ifprof
\begin{corrige}
\end{corrige}
\else
\fi

Refaire l'étude fréquentielle permettant de retrouver le diagramme de Bode document réponse.


\subparagraph{}
\textit{Déterminer à l'aide de la simulation la valeur du coefficient de surtension $Q$ ainsi que la pulsation  $\omega_r$ correspondante En déduire la vitesse $\dot{y}_0$ de l'avion en km par h pour cette pulsation $\omega_r$.}
\ifprof
\begin{corrige}
\end{corrige}
\else
\fi


\subparagraph{}
\textit{Déterminer les expressions de $z_p(t)$ et de $z_a(t)$ pour la pulsation $\omega_r$. Construire à l’aide d’un tableur les
courbes de $z_p(t)$ et de $z_a(t)$. Vérifier les expressions de $z_p(t)$ et de $z_a(t)$ en comparant les courbes obtenues par Xcos et celles obtenues par le tableur.}
\ifprof
\begin{corrige}
\end{corrige}
\else
\fi

\subsection*{Étude des influences des irrégularités de la piste sur le confort des passagers}


L'analyse du confort des passagers ou des sollicitations structurelles de l'avion 
est conduite à partir de l'examen 
de la fonction de transfert liant 
l'accélération verticale $\gamma_a(t)$
de l'avion à la hauteur des aspérités de la piste.


On donne la fonction de transfert $\dfrac{\gamma_a(p)}{Z_p(p)}=H_2(p) H_3(p)$ où : $H_2(p)=\dfrac{1+0,2 p}{\left(1+0,1 p \right)^2}$, $H_3(p)=\dfrac{p^2}{1+\dfrac{2\xi}{\omega_1}p+\dfrac{1}{\omega_1^2}p^2}$ avec $\omega_1=\SI{300}{rad/s}$ et $\xi=0,1$. 



\subparagraph{}
\textit{Modifier le schéma-blocs
précédent afin d’ajouter la fonction de transfert $H_3(p)$ puis tracer le diagramme
de Bode pour un domaine fréquentiel allant de \SI{0,1}{rad/s} à \SI{10000}{rad/s}. Mettre en forme les courbes résultats
dans Xcos pour retrouver le diagramme document réponse 2.}
\ifprof
\begin{corrige}
\end{corrige}
\else
\fi


\subparagraph{}
\textit{Tracer les points correspondants à $\omega=\SI{1}{rad/s}$ et $\omega=\SI{1800}{rad/s}$ sur la courbe de phase du document réponse 2 ainsi que les asymptotes lorsque $\omega$ tend vers 0 et $\omega$ tend vers l'infini sur les courbes de gain et de phase.}
\ifprof
\begin{corrige}
\end{corrige}
\else
\fi


La norme NF E904012 propose un modèle de tolérance physiologique aux vibrations verticales. Le graphe extrait de la norme permet d’identifier une << zone de mal des transports >> (zone A) et une << zone d’inconfort vibratoire >> (zone B).

\subparagraph{}
\textit{Identifier sur l’abaque l’accélération 
verticale maximale $\gamma_{a0}$
que peut supporter le corps humain 
sollicité avec une fréquence comprise 
entre \SI{4}{Hz} et \SI{8}{Hz} 
pendant \SI{30}{min} sans être accommodé.
}
\ifprof
\begin{corrige}
\end{corrige}
\else
\fi


\subparagraph{}
\textit{Déterminer le rapport d’amplitude maximal $\dfrac{\gamma_{a0}}{z_{p0}}$
en décibel pour une amplitude d’aspérités de la piste de \SI{1}{cm}.}
\ifprof
\begin{corrige}
\end{corrige}
\else
\fi


\subparagraph{}
\textit{Conclure sur le confort des passagers pour la plage de fréquence \SI{4}{Hz} à \SI{8}{Hz}.}
\ifprof
\begin{corrige}
\end{corrige}
\else
\fi


\begin{center}
\includegraphics[width=\linewidth]{images/img_07}
\end{center}


\ifprof
\else
\end{multicols}
\fi


\begin{center}
\includegraphics[width=\linewidth]{images/img_08}
\includegraphics[width=\linewidth]{images/img_09}
\end{center}

\end{document}






\subparagraph{}
\textit{}
\ifprof
\begin{corrige}
\end{corrige}
\else
\fi



\subparagraph{}
\textit{}
\ifprof
\begin{corrige}
\end{corrige}
\else
\fi



\subparagraph{}
\textit{}
\ifprof
\begin{corrige}
\end{corrige}
\else
\fi



\subparagraph{}
\textit{}
\ifprof
\begin{corrige}
\end{corrige}
\else
\fi

\ifprof
\else
\end{multicols}
\fi

\end{document}


\subparagraph{}
\textit{}
\ifprof
\begin{corrige}
\end{corrige}
\else
\fi
