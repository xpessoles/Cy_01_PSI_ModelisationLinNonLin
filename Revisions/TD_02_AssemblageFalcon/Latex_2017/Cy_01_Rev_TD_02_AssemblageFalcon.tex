\documentclass[10pt,fleqn]{article} % Default font size and left-justified equations
\usepackage[%
    pdftitle={Modélisation systèmes multiphysiques : définitions et performances},
    pdfauthor={Xavier Pessoles}]{hyperref}

\input{style/new_style}
\input{style/macros_SII}

\fichetrue
%\fichefalse

\proftrue
%\proffalse

\tdtrue
%\tdfalse

%\courstrue
\coursfalse

%%%%%%%%%%%%%%%%%%%
\usepackage{numprint}
\usetikzlibrary{calc}
%definition style pour mettre un fond blanc dans un node sans avoir des marges énormes
\tikzset{fondblanc/.style={ inner sep=2pt,fill=white,outer sep = 5pt}} 
\tikzset{fondblanc2/.style={ inner sep=2pt,fill=white}} 
\tikzset{fondblanc3/.style={ inner sep=1pt,fill=white,outer sep = 2pt}} 
\usetikzlibrary{calc,circuits.ee.IEC}
\usetikzlibrary{shapes}
\usepackage[european resistor, european voltage, european current]{circuitikz}
\usetikzlibrary{babel}
\usepackage{standalone}
\standaloneconfig{mode=buildnew}

\usepackage{siunitx}

\usepackage{picins}

% -------------------------------------
% Déclaration des titres
% -------------------------------------

\def\discipline{Sciences \\Industrielles de \\ l'Ingénieur}
\def\xxtete{Sciences Industrielles de l'Ingénieur}

\def\classe{Cy01 - Rév}
\def\xxnumpartie{Cycle 01}
\def\xxpartie{Modéliser le comportement linéaire et non linéaire des systèmes multiphysiques}

\def\xxnumchapitre{Révisions \vspace{.2cm}}
\def\xxchapitre{\hspace{.12cm}}% Définitions préliminaires et détermination des performances}

\def\xxposongletx{2}
\def\xxposonglettext{1.45}
\def\xxposonglety{19}%16

\def\xxonglet{Cycle 01 -- Rév 1}

\def\xxactivite{TD 02}
\def\xxauteur{\textsl{Xavier Pessoles} \\ \textsl{PSI -- MP -- Lycée La Martinière Monplaisir Lyon}}


\def\xxtitreexo{Cellule d'assemblage pour avion Falcon}
\def\xxsourceexo{D'après concours E3A -- PSI 2015.}

\def\xxcompetences{%
\textsl{%
\textbf{Savoirs et compétences :}\\
}}

\def\xxfigures{
\includegraphics[width=.8\textwidth]{images/image1}
}%figues de la page de garde

\def\xxpied{%
Cycle 01 -- Modéliser le comportement des systèmes multiphysiques\\
Révisions %1 -- \xxactivite%
}

\setcounter{secnumdepth}{5}
%---------------------------------------------------------------------------


\begin{document}
%\chapterimage{png/Fond_Cin}
\input{style/new_pagegarde}
\vspace{7cm}
\pagestyle{fancy}
\thispagestyle{plain}

\setlength{\columnseprule}{0.4pt}
\begin{multicols}{2}
\section*{Mise en situation}
\subsection*{Présentation}
Le tronçon central du fuselage du Falcon 7X est assemblé par rivetage grâce à un robot 6 axes. Les rivets sont stockés dans des cassettes rangées verticalement. Un chariot de sélection se déplace verticalement pour déplacer une buse d’aspiration qui permettra d’acheminer les rivets contenus dans la cassette vers l’effecteur (robot). Le chariot fait l’objet de cette étude.

\begin{center}
\includegraphics[width=7cm]{images/image5}
\end{center} 

 
L'objectif de cette partie est de valider les choix effectués par la société pour le sous ensemble de sélection des fixations de la cellule (exigence 1.1).

\vfill

\begin{center}
\includegraphics[width=7cm]{images/image6}
\end{center} 




 
\subsection*{Axe chariot}
Le déplacement du chariot est assuré par un axe numérique asservi en vitesse et en position. Cet axe est composé d’un moteur à courant continu, d’un système de transmission de puissance de type poulies / courroie et d’un rail.
 
\begin{center}
\includegraphics[width=7cm]{images/image7}
\end{center} 


\subsection*{Modélisation du système de déplacement du chariot}

\begin{center}
\includegraphics[width=7cm]{images/image8}
\end{center} 







\section*{Sélectionner les fixations – Exigence 1.1}
Afin de sélectionner le type de fixation, la buse d’aspiration doit être déplacée en face de la cassette avec une erreur inférieure à $\SI{0,5}{mm}$ (voir exigences fonctionnelles). Cependant le fabricant du système poulie-courroie du rail indique déjà une erreur de $\pm \SI{0,25}{mm}$ due notamment à l’élasticité de la courroie. Par conséquent, l’erreur en position de la commande doit être nulle. De plus, afin de ne pas perdre de temps lors de la production, le temps maximal de déplacement lors de la sélection est imposé à une seconde.

L’étude se fera dans le cas le plus défavorable c'est-à-dire un déplacement du chariot vers le haut entre les deux cassettes de rivets les plus éloignées. L’axe de déplacement est appelé $\vect{y_c}$
\subsection*{Notations domaine temporel – domaine de Laplace}

Les notations entre le domaine temporel et celui de Laplace sont données dans la suite. Ainsi, si la fonction $f(t)$ possède une transformée de Laplace, elle sera notée : $F(p) = \mathcal{L}[f(t)]$.
Les équations caractéristiques du moteur à courant continu sont rappelées ci-dessous (les conditions de Heaviside sont respectées) :
\begin{itemize}[label=\ding{112},font=\color{ocre}] 
\item $u(t)=e(t)+L \dfrac{di(t)}{\text{d} t}+Ri(t)$;
\item $e(t)=K_E \omega_m (t)$ ;
\item $C_M (t)=K_C i(t)$ ;
\item $J_{eq}  \dfrac{d\omega_m)}{\text{d} t}+f\omega_m (t)=C_M (t)-C_R (t)$.
\end{itemize}

Avec : 
\begin{itemize}[label=\ding{112},font=\color{ocre}] 
\item $u(t)$ : tension moteur ;
\item $i(t)$ : courant moteur ; 
\item $e(t)$ : force contre-électromotrice ;
\item $\omega_m(t)$ : vitesse de rotation moteur ;
\item $C_M (t)$ : couple moteur ;
\item $C_R (t)$: couple résistant modélisant l’action de pesanteur.
\end{itemize}

\subsection*{Critères à respecter pour l’exigence 1.2}

\footnotesize{\begin{center}
\begin{tabular}{|p{2.5cm}|p{2.5cm}|p{2.5cm}|}
\hline
Exigence	& Critères & Niveaux \\ \hline
Déplacer le chariot	& Précision :
erreur statique par rapport à une consigne de vitesse constante	
& NULLE \\ \hline
	& Rapidité : temps de réponse à 5\% en réponse à une consigne échelon 
	& $Tr_{5\%} = \SI{0,1}{s}$  maxi \\ \hline
	& Stabilité : & \\
	& Marge de gain & \SI{6}{dB} mini \\
	&Marge de gain & $\ang{45}$ mini \\
\hline
\end{tabular}
\end{center}}


\subsection*{Choix d’une architecture de la chaine de transmission}
\subparagraph{}
\textit{Proposer sous la forme d’un schéma une autre solution permettant le déplacement du chariot. La conversion de l’énergie électrique en énergie mécanique par un moteur doit être conservée.}

Compte tenu des vitesses de translation importantes, le système retenu est de type poulie-courroie.

\subsection*{Détermination de l’inertie équivalente} 
Les grandeurs caractéristiques (notations et valeurs) des éléments de l’axe du chariot sont données dans le tableau ci-dessous :
\begin{center}
\begin{tabular}{|p{3cm}|c|c|}
\hline
Moment d’inertie du rotor du moteur autour de son axe&	$J_m$ & $\num{140d-6}\si{kg.m^2}$ \\ \hline
Moment d’inertie du réducteur ramené à l’arbre moteur&	$J_{réd}$ & $\num{60d-4}\si{kg.m^2}$ \\ \hline
Moment d’inertie de la poulie motrice autour de son axe&	$J_{PM}$	&$ \num{38d-4}\si{kg.m^2}$ \\ \hline
Moment d’inertie de la poulie réceptrice autour de son axe&	$J_{PR}$ & $\num{38d-4}\si{kg.m^2}$ \\ \hline
Masse totale du chariot	&$M$ &$\SI{5}{kg}$ \\ \hline
Vitesse de rotation de l’arbre moteur &$\omega_m$ &  \\ \hline
Vitesse de rotation de l’arbre de sortie du réducteur	&$\omega_r$&  \\ \hline
Rayon d’une poulie motrice ou réceptrice	& $R_P$ &$\SI{45}{mm}$ \\ \hline
Rapport de réduction réducteur ($\omega_r/\omega_m$)	& $\lambda$	&$1/5$ \\ \hline
\end{tabular}
\end{center}

\subparagraph{}
\textit{À partir des grandeurs définies déterminer l’expression littérale de l’inertie équivalente $J_{eq}$ de l’ensemble $\Sigma=\{moteur+réducteur+poulies+chariot\}$ ramenée sur l’arbre moteur. Cette inertie équivalente est définie par $E_c (\Sigma)=1/2 J_{eq} \omega_m^2$.}

\subparagraph{}
\textit{Déterminer la valeur numérique de l’expression précédente.}

\subsection*{Modèle de connaissance du moteur à courant continu}
\begin{obj}
L’objectif de cette partie est d’établir un modèle de la motorisation de l’axe afin de simuler un déplacement.
\end{obj}

\subparagraph{}
\textit{À partir des équations du moteur à courant continu, réaliser le schéma bloc du moteur à courant continu.}

\subparagraph{}
\textit{En considérant que $C_R (p)=0$, déterminer la fonction de transfert $H_M (p)=\dfrac{\Omega_m (p)}{U(p)}$ sous sa forme canonique.}

\subparagraph{}
\textit{Montrer que la fonction de transfert $H_M (p)$ peut se mettre sous la forme $H_M (p)=\dfrac{K_C}{K_C K_e+RJ_{eq} p+LJ_{eq} p^2 }$. Justifier la réponse. Pour cette question, la valeur numérique de $J_{eq}$ considérée sera $J_{eq}=\num{7d-3}\si{kg.m^2}$ indépendamment du résultat numérique calculé précédemment.}
\end{multicols}
\end{document}
\subparagraph{}
\textit{Montrer qu’avec l’expression, H_M (p) peut s’écrire sous la forme H_M (p)=K_M/(1+T_E p)(1+T_M p)  avec T_E<T_M.}

\section*{Étude de l’asservissement en position de l’axe}
\subsection*{Modélisation de l’asservissement en position}
La partie précédente a permis de déterminer un modèle du moteur. La suite de l’étude va permettre, par simulation, de déterminer les réglages nécessaires de l’axe vis-à-vis du cahier des charges. La figure suivante présente le principe de l’asservissement de l’axe du chariot.
 

Les grandeurs caractéristiques des blocs de l’asservissement de l’axe chariot sont données dans le tableau ci-dessous :
Générateur	K_G	À déterminer
Capteur de position	K_capt	5⋅〖10〗^(-3) V⋅rad^(-1)
Correcteur amplificateur	C(p)	Variable
Quelle doit être la valeur de K_G pour assurer un asservissement correct (c’est à dire l’écart ε doit être nul si la position de l’axe est identique à la consigne) ?
Donner le schéma bloc de l’asservissement.

Étude du modèle simplifié
Afin de faciliter les calculs, le schéma bloc à retour unitaire est donné figure suivante. Le couple résistant C_R dû à l’action de pesanteur est supposé constant.
 
Avec :
	H_M (p)=K_M/(1+T_E p)(1+T_M p) ;
	H_C (p)=(((R+Lp) K_M)/K_C )/(1+T_E p)(1+T_M p) ;
	C_R (p)=C_r/p  ;
	K_R=R_p λ
Donner l’expression de Y(p).
On souhaite déterminer l’erreur en position du système. Calculer l’écart statique pour C(p)=K_p puis C(p)=Ki/p..
On souhaite que lorsque le système se déplace à vitesse constante, l’erreur sur la vitesse atteinte par le système soit nulle. Quelle sollicitation doit-on utiliser. Calculer l’écart statique pour C(p)=K_p puis C(p)=Ki/p.
Conclure.

Afin de répondre totalement au cahier des charges, l’utilisation d’un correcteur proportionnel intégral dérivé est retenue. En effet, la commande de l’axe intègre directement ce type de correcteur. Dans la suite du problème, le correcteur C(p) sera de la forme : C(p)=K_I (1+1/(T_I p))(1+T_D p). Le réglage des coefficients a été fait par simulation numérique.
Afin de vérifier maintenant le critère de rapidité, on donne la réponse temporelle (figure page suivante) de l’axe à un échelon de position de 1 m.

Conclure sur la conformité au cahier des charges du système ainsi réglé. 
 


 

 

Tracé de diagramme de Bode
On considère C_R (p)=0. On prendra K_M=0,8 rad⋅s^(-1)⋅V^(-1), T_e=0,0051 s,T_m=0,0074 s.
Tracer le diagramme de Bode de la fonction de transfert en boucle ouverte pour C(p)=1. Déterminer les marges de phase et les marges de gain. 
Tracer le diagramme de Bode de la fonction de transfert en boucle ouverte pour C(p)=1/p. Déterminer les marges de phase et les marges de gain.
On donne ci-dessous les diagrammes de Bode avec les correcteurs optimisés. Déterminer les marges de gain et marges de phase. 
Vérification des performances de l’axe du magasin de rivets
Afin de vérifier les réglages précédents, un essai sur le système réel est réalisé. Une consigne de 2 m est donnée. L’absence de système d’acquisition dédié impose un système de mesure extérieur au système réel. C’est un dispositif d’analyse d’image qui est retenu pour ces mesures.

À partir des relevés ci-dessous, conclure sur le respect des exigences fonctionnelles de l’axe du magasin de stockage des rivets (Exigence 1.1).

 

 

 	 
Position du chariot en fonction du temps	Position et vitesse du chariot en fonction du temps




\end{document}


