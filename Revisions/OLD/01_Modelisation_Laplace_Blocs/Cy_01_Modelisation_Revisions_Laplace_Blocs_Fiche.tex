\documentclass[10pt,fleqn]{article} % Default font size and left-justified equations
\usepackage[%
    pdftitle={SLCI : Modélisation des SLCI par transformée de Laplace et schémas blocs},
    pdfauthor={Xavier Pessoles}]{hyperref}

\input{style/new_style}
\input{style/macros_SII}

\fichetrue
%\fichefalse

\proftrue
%\proffalse

%\tdtrue
\tdfalse

%\courstrue
\coursfalse

%%%%%%%%%%%%%%%%%%%
\usepackage{etex}
\usepackage{schemabloc}
%\usepackage{rpcinematik}
%\usepackage{bodegraph}
\usepackage{booktabs}
\usepackage[hang,raggedright]{subfigure}
\usepackage{numprint}

\usetikzlibrary{calc}
\usepackage{multirow} %fusions des lignes dans tableau
\usepackage{longtable}

\usepackage{picins}    %permet d'insérer une image à coté d'un texte \parpic[r]\includegraphics{}}texte...
\usepackage{placeins}

%definition style pour mettre un fond blanc dans un node sans avoir des marges énormes
\tikzset{fondblanc/.style={ inner sep=2pt,fill=white,outer sep = 5pt}} 
\tikzset{fondblanc2/.style={ inner sep=2pt,fill=white}} 
\tikzset{fondblanc3/.style={ inner sep=1pt,fill=white,outer sep = 2pt}} 
\usetikzlibrary{calc,circuits.ee.IEC}
\usetikzlibrary{shapes}
\usepackage[european resistor, european voltage, european current]{circuitikz}
\usetikzlibrary{babel}

\newlength{\currentparskip}

\usepackage{wasysym}

\usepackage{relsize}
%Définition des torseurs :
%\newcommand{\torseur}[2]{\left \{\mbox{\relsize{2}{$\mathcal {#1}$}\relsize{-2}}\phantom{}_{\mbox{\scriptsize $#2$}} \right \}}
\newcommand{\torseurh}[3]{\left\{\genfrac{}{}{0pt}{0}{#1}{#2}\right\}_{#3}}
\newcommand{\torseurv}[8]{
\left\{\begin{array}{@{}c|c@{}} #1 & #4 \\ #2 & #5 \\ #3 & #6 \end{array} \right\}_{#7,#8}
}
\newcommand{\torseurrdm}[2]{\left \{\mbox{\relsize{2}{$\mathcal {#1}$}\relsize{-2}}{\scriptsize\mbox{$#2$}} \right \}}
\newcommand{\torseurhrdm}[3]{\left\{\genfrac{}{}{0pt}{0}{#1}{#2}\right\}_{#3}}
\newcommand{\resultanterdm}[2]{\vv {#1}{{#2}}}

\newcommand{\derivee}[2]{\left.\dfrac{\d #1}{\d t}\right|_{#2}}
\newcommand{\tripleint}{\int\!\!\!\!\!\int\!\!\!\!\!\int}

% Notation cinématique et statique
\newcommand{\cinematique}[2]{\mbox{#1}/\mbox{#2}}
\newcommand{\statique}[2]{\mbox{#1}\rightarrow\mbox{#2}}
\newcommand{\moment}[3]{\vv {#1}_{\scriptsize{#3}}(#2)}
\newcommand{\resultante}[2]{\vv {#1}_{\scriptsize{#2}}}

%Commande de base
\newcommand{\jo}{\left(j\omega\right)} % j \omega dans l'analyse fréquentielle
\newcommand{\tl}{\xrightarrow{\mathcal{L}}} % transformée de laplace sur fleche
\newcommand{\tli}{\xrightarrow{\mathcal{L}^{-1}}} % transformée inverse de laplace sur fleche
\newcommand{\Laplace}{\mathcal{L}} % transformée de laplace

\renewcommand{\d}[1][]{\mathrm{d#1}}
\newcommand{\dd}[1][]{\mathrm{d#1}}
%\newcommand{\vect}[2]{{#1}\wedge{#2}}
%\newcommand{\base}[3]{(\vec #1,\vec #2,\vec #3)}
%\newcommand{\vectbase}[4]{{\vphantom{\left| \begin{matrix}
%#1\\#2\\#3 \end{matrix} \right|}}_{#4}{\left| \begin{matrix}
%#1\\#2\\#3 \end{matrix} \right.}}

%\newcommand{\unitesi}[1]{\,\mathrm{#1}} % Permet d'écrire une unité en chiffre romain en mode math précédé de l'espace moyenne insécable.

%%%%%%%%%%%%%%%%



% -------------------------------------
% Déclaration des titres
% -------------------------------------

\def\discipline{Sciences \\Industrielles de \\ l'Ingénieur}
\def\xxtete{Sciences Industrielles de l'Ingénieur}

\def\classe{Cy01 - R 01}
\def\xxnumpartie{Cycle 01}
\def\xxpartie{Modéliser le comportement linéaire et non linéaire des systèmes multiphysiques}

\def\xxnumchapitre{Révions 1 \vspace{.2cm}}
\def\xxchapitre{\hspace{.12cm} Modélisation par transformée de Laplace et par Schéma-blocs}

\def\xxposongletx{2}
\def\xxposonglettext{1.45}
\def\xxposonglety{19}%16

\def\xxonglet{Cycle 01 -- Rév 1}

\def\xxactivite{Fiche}
\def\xxauteur{\textsl{Xavier Pessoles}}

\def\xxcompetences{%
\textsl{%
\textbf{Savoirs et compétences :}\\
}}

\def\xxfigures{
%incgraphics[width=.8\textwidth]{}%images/prot_01}
}%figues de la page de garde

\def\xxpied{%
Cycle 01 -- Modélisation des systèmes linéaires et non linéaires\\
Révision 1 -- \xxactivite%
}

\setcounter{secnumdepth}{5}
%---------------------------------------------------------------------------


\begin{document}
%\chapterimage{png/Fond_Cin}
\input{style/new_pagegarde}
\vspace{2cm}
\pagestyle{fancy}
\thispagestyle{plain}

\section{Définitions}



\begin{defi}[Informations analogiques et numériques] ~\\
\begin{itemize}[label=\ding{112},font=\color{ocre}] 
\item Une information analogique peut prendre, de manière continue, toutes les valeurs
possibles dans un intervalle donné. Un signal analogique peut être représenté
par une courbe continue. Les grandeurs physiques (température, vitesse,
position, tension, ...) sont des informations analogiques.

\item Une information numérique sous la forme d'un mot binaire est constituée de
plusieurs bits (variables binaires 0/1). Cette information numérique est en
général issue d'un traitement (échantillonnage et codage) d'une information
analogique. On parle de conversion analogique numérique (CAN).
\end{itemize}
\end{defi}

\begin{defi}[Systèmes automatiques ou asservis]

Un système asservi est commandé par \textbf{une (ou des) entrée(s)} qu'il
transforme en \textbf{grandeur(s) de sortie}.
Les entrées sont de deux types : 
\begin{itemize}
 \item la loi de consigne $e(t)$ est une grandeur de commande qui est
modifiable;
\item la perturbation : c'est une entrée parasite qui nuit au bon
fonctionnement du système. On ne peut pas modifier les perturbations.
\end{itemize}

La sortie $s(t)$ est une grandeur \textbf{observable} (par des capteurs) qui
permet de juger de la qualité de la tâche accomplie.
\end{defi} 


\begin{defi}[Systèmes suiveurs et régulateurs] ~\\
\begin{itemize}[label=\ding{112},font=\color{ocre}] 
\item Pour un système suiveur la consigne $e(t)$ fluctue au cours du temps. Le système doit faire son possible pour qu'à chaque instant la cible soit suivie.
\item Pour un système régulateur la consigne $e(t)$ est constante. Les perturbations font varier la position du système. Il doit donc de façon automatique revenir à la position commandée.
\end{itemize}
\end{defi}



\subsection*{Performance des systèmes -- Critères graphiques}
\begin{defi}[Écart statique $\varepsilon_S$ -- Écart dynamique $\varepsilon_V$]

Le système est piloté par une valeur constante. On définit alors l'écart statique $\varepsilon_S$ comme l'écart entre la consigne fixe et la réponse $s(t)$ en régime permanent.
\end{defi}

\begin{defi}[Précision -- Écart dynamique  $\varepsilon_V$]

Encore appelé écart de traînage ou écart de poursuite, il représente la différence entre la consigne variable et la réponse en régime permanent. 
\end{defi}

\begin{defi}[Rapidité]

La rapidité est caractérisée par le temps que met le système à réagir à une
variation brusque de la grandeur d'entrée (temps de réponse). Cette notion est
fortement liée à la notion de précision dynamique.
\end{defi}

\begin{methode}
[Détermination du temps de réponse à $n\%$] (En pratique $n=5$).\\

\begin{enumerate}
 \item Tracer sur le même graphe la consigne $e(t)$ et la réponse du système
$s(t)$.
\item Tracer la droite correspondant à la valeur asymptotique de $s(t)$.
\item Tracer la bande correspondant à une variation de $\pm n\%$ de la valeur
asymptotique.
\item Relever la dernière valeur à partir de laquelle $s(t)$ coupe la bande et
n'en sort plus.
\end{enumerate}
\end{methode}

\begin{defi}[Stabilité]

La stabilité traduit la propriété de convergence temporelle asymptotique vers
un état d'équilibre. 
\end{defi}


\begin{defi}[Transformée de Laplace] 

\end{defi}


\begin{theorem}[Théorèmes généraux] ~\\
\begin{itemize}[label=\ding{112},font=\color{ocre}] 
\item Théorème de la valeur initiale :
\item Théorème de la valeur finale
\end{itemize}
\end{theorem}


\begin{prop}
\textbf{Propriétés des sections droites}
On considère une poutre de section droite $S$ et d'axe $\left(G,\vect{x} \right)$. On note :
\begin{itemize}
\item moment quadratique de $S$ par rapport à $\left(G,\vect{y} \right)$ : $I_{Gy} = \iint\limits_S z^2 \text{d}S$;
\item moment quadratique de $S$ par rapport à $\left(G,\vect{z} \right)$ : $I_{Gz} = \iint\limits_S y^2 \text{d}S$;
\item moment polaire de $S$ par rapport à $\left(G,\vect{x} \right)$ : $I_{Gx} = \iint\limits_S \left( y^2 + z^2 \right) \text{d}S$;
\item $I_{Gx}=I_{Gy}+I_{Gz}$.
\end{itemize}
\end{prop}

\begin{theorem}
\textbf{Théorème de Huygens} ~\\
On a, avec $\vect{AG}=\left(a,b,c \right)$ :
$$
I_{Ay}=I_{Gy} + Sc^2 
\quad 
I_{Az}=I_{Gz} + Sb^2 
\quad 
I_{Ax}=I_{Gx} + S\left(b^2+c^2 \right). 
$$
\end{theorem}

\subsection*{Cas du cisaillement}

\begin{defi} -- 
\textbf{Contrainte et déformations} ~\\

Pour une section cisaillée, la contrainte tangentielle est de la forme : 
$$
\tau = \dfrac{T_y}{S} \quad \text{avec } S \text{ section cisaillée.}
$$
\end{defi}


\begin{resultat}
\textbf{Dimensionnement d'une pièce cisaillée} ~\\
On a : $|\tau| < \dfrac{Rg}{s}$ avec $s$ coefficient de sécurité et $Rg$ limite au glissement tel que $Rg = \xi Re$ :
\begin{itemize}
\item acier doux (\% C < 0,2) : $\xi =0,5$;
\item acier mi-doux (\% C compris entre 0,2 et 0,32) : $\xi =0,6$;
\item acier mi-durs (\% C compris entre 0,32 et 0,45) : $\xi =0,7$;
\item acier durs (\% C>0,45) : $\xi =0,8$;
\item fontes (\% C>1,7) : $\xi \in [0,77;1]$.
\end{itemize}

\end{resultat}

\includestandalone{images/image_13}
\end{document}


