\documentclass[10pt,fleqn]{article} % Default font size and left-justified equations
\usepackage[%
    pdftitle={Modélisation SLCI : Rapidité des systèmes},
    pdfauthor={Xavier Pessoles}]{hyperref}
    
\input{style/new_style}
\input{style/macros_SII}
\usepackage{multicol}
\usepackage{siunitx}
%\usepackage{picins}

\fichetrue
%\fichefalse

\proftrue
\proffalse

\tdtrue
%\tdfalse

\courstrue
\coursfalse

\newif\ifnormal
\normaltrue
%\normalfalse

\newif\ifdifficile
\difficilefalse
%\difficiletrue

\newif\iftdifficile
\tdifficilefalse
%\tdifficiletrue

\newif\ifcolle
\colletrue
%\normalfalse


\def\discipline{Sciences \\Industrielles de \\ l'Ingénieur}
\def\xxtete{Sciences Industrielles de l'Ingénieur}

\def\classe{PSI$\star$ -- MP}
\def\xxnumpartie{Cycle 02}
\def\xxpartie{Modéliser les systèmes asservis dans le but de prévoir leur comportement}


\def\xxnumchapitre{Chapitre 2 \vspace{.2cm}}
\def\xxchapitre{\hspace{.12cm} Rapidité des systèmes}


\def\xxtitreexo{Station d'épuration}
\def\xxsourceexo{\hspace{.2cm} \footnotesize{CCP MP 2012}}


\def\xxposongletx{2}
\def\xxposonglettext{1.45}
\def\xxposonglety{20}
%\def\xxonglet{Part. 1 -- Ch. 3}
\def\xxonglet{Cycle 02}

\def\xxactivite{Colle 6}
\def\xxauteur{\textsl{Xavier Pessoles}}

\def\xxcompetences{%
\textsl{%
\textbf{Savoirs et compétences :}\\
%Les sources sont associées par un \emph{hacheur série}. La détermination des grandeurs électriques associées à ce montage permet de conclure vis à vis du cahier des charges.
%\noindent \textbf{Résoudre :} à partir des modèles retenus :
%\begin{itemize}[label=\ding{112},font=\color{ocre}] 
%\item choisir une méthode de résolution analytique, graphique, numérique;
%\item mettre en \oe{}uvre une méthode de résolution.
%\end{itemize}
%\begin{itemize}[label=\ding{112},font=\color{ocre}] 
%\item \textit{Rés -- C1.1 :} Loi entrée sortie géométrique et cinématique -- Fermeture géométrique.
%\end{itemize}
%
%\noindent \textit{Mod2 -- C4.1 :} Représentation par schéma bloc.
}}

\def\xxfigures{
\includegraphics[width=.5\linewidth]{images/fig_01}
}%figues de la page de garde


\def\xxpied{%
Cycle 01 -- Modéliser le comportement des systèmes multiphysiques\\
Chapitre 2 -- \xxactivite%
}

\setcounter{secnumdepth}{5}
%---------------------------------------------------------------------------


\begin{document}
%\chapterimage{png/Fond_Cin}
\input{style/new_pagegarde}
\vspace{4.5cm}
\pagestyle{fancy}
\thispagestyle{plain}


\def\columnseprulecolor{\color{ocre}}
\setlength{\columnseprule}{0.4pt} 

\ifprof
%\begin{multicols}{2}
\else
\begin{multicols}{2}
\fi

\section*{Présentation}


\ifcolle
\else
Les boues sont constituées d’eau et de matière sèche. La siccité est le pourcentage massique de matière
sèche. Ainsi, une boue avec une siccité de 10\,\% présente une humidité de 90\,\%. Afin
d’incinérer les boues, il faut les déshydrater pour atteindre une siccité de 20\,\%. La déshydratation
mécanique par centrifugation permet de séparer l’eau des matières sèches dans les boues.
La centrifugation se base sur la différence de densité entre les matières sèches et l’eau présente dans
cette boue (figure 10). La boue arrive avec une certaine vitesse horizontale par un coté de la centrifugeuse
(arrivée de boue liquide sur la figure 10). L’eau traverse alors toute la centrifugeuse dans sa
zone centrale tandis que les matières en suspension sont plaquées contre le tambour extérieur du fait
de sa vitesse de rotation. Une vis intérieure, tournant dans le même sens que le tambour mais à une
vitesse plus importante, vient alors récupérer les boues et les évacuer en sens inverse de l’eau jusqu’à
la sortie latérale (sortie de boue déshydratée sur la figure 10). La compréhension du fonctionnement
des flux d’eau et de boue dans la centrifugeuse n’est pas nécessaire à la suite de l’étude.


\begin{center}
\includegraphics[width=\linewidth]{images/fig_01}
%\textit{Fig 10}
\end{center}

\fi

\begin{obj}
L’objectif de cette partie est de vérifier le contrôle de la siccité des boues pour leur incinération
(FS5).
\end{obj}

\begin{center}
\includegraphics[width=\linewidth]{images/fig_07}
%\textit{Fig 11}
\end{center}


\ifcolle
\else 
La boue visqueuse est cisaillée par la différence de vitesse entre la vis et le tambour (bol extérieur).
La siccité de la boue est directement liée à cette différence de vitesse dont
l’asservissement est l’objet de l’étude suivante. La chaîne cinématique est représentée sur la figure
11.

La séquence de lancement de la centrifugeuse se déroule en trois phases :
\begin{itemize}
\item mise en marche du premier moteur $M_{\text{tambour}}$ jusqu’à ce que le tambour 1 atteigne sa vitesse
de consigne de \SI{2 000}{tours/min}. Le moteur $M_{\text{rel}}$ est à l’arrêt;
\item mise en marche du deuxième moteur $M_{\text{rel}}$ jusqu’à ce que la vitesse différentielle de
\SI{2}{tours/min} soit atteinte entre le tambour 1 et la vis 3. La vis 3 tourne ainsi plus vite que le
tambour 1;
\item la boue liquide est ensuite introduite.
\end{itemize}

Notation : La vitesse de rotation du solide $i$ par rapport au solide $j$ est notée $\omega_{ij}$ (rad/s). Elle sera
notée $N_{ij}$ lorsqu’elle est exprimée en tours/min. Le nombre de dents d’un engrenage « $i$ » est noté $Z_i$.


\begin{center}
\includegraphics[width=\linewidth]{images/fig_02}
%\textit{Fig 11}
\end{center}

\fi

La structure de l’asservissement est donnée ci-dessous. La consigne en vitesse du tambour est une
constante : $N_{10c}(t)=\SI{2000}{tours/min}$.

\begin{center}
\includegraphics[width=\linewidth]{images/fig_03}
%\textit{Fig 11}
\end{center}

\begin{center}
\includegraphics[width=\linewidth]{images/fig_04}
%\textit{Fig 11}
\end{center}



\subparagraph{}\textit{Mettre le schéma de la figure précédente sous la forme de la figure suivante. Donner les expressions sous forme canonique de $H_1(p)$ et de $H_2(p)$ en fonction des données du moteur $M_{\text{tambour}}$.}
\ifprof
\begin{corrige}
\end{corrige}
\else
\fi

\begin{center}
\includegraphics[width=\linewidth]{images/fig_05}
%\textit{Fig 11}
\end{center}

Pour la suite du sujet, vous prendrez : $K_1 =\SI{71,4}{(tours/min)/(N.m)}$; $K_2=5,1 \times 10^{-3}$; $\tau_2 = \SI{20,2}{s}$.


\subparagraph{}\textit{Exprimer l'écart $\varepsilon(p)$ (écart en sortie du second comparateur) en fonction de $C_r(p)$ et $N_{10c}(p)$.}
\ifprof
\begin{corrige}
\end{corrige}
\else
\fi

\subparagraph{}\textit{Pour les 4 cas du tableau ci-dessous en fonction de $K_1$, $K_2$, $\tau_2$, $K_c$, $K_i$, $N_{10}c (t)$ et $C_r (t)$ . Pour quel(s) correcteur(s), le critère de précision de la fonction FS5 est-il vérifié ?}
\ifprof
\begin{corrige}
\end{corrige}
\else
\fi

\begin{center}
\begin{tabular}{|p{3cm}|c|c|}
\hline
$\varepsilon_S$ & $C_T(p)=K_c$ & $C_T(p)=K_c  + \dfrac{K_i}{p}$ \\
\hline
$N_{10c}(t)=\SI{2000}{tr/min}$ et $C_r(t)=0$ &  & \\
\hline
$N_{10c}(t)=\SI{0}{tr/min}$ et $C_r(t)=\SI{3}{kNm}$ &  & \\
\hline
\end{tabular}
\end{center}

%\begin{center}
%\includegraphics[width=\linewidth]{images/fig_06}
%%\textit{Fig 11}
%\end{center}


\subsection*{Réglage du correcteur PI}

On choisit d’installer un correcteur « PI » de vérifier le cahier des charges : $C_T(p)=K_C + \dfrac{K_i}{p}$.
Le réglage se fera en prenant le couple résistant nul et on notera $\dfrac{N_{10}(p)}{N_{10c}(p)}=H(p)$.

Les critères prépondérants de précision et de stabilité étant vérifiés, il reste à régler les deux paramètres
$K_c$ et $K_i$ à partir des critères de dépassement et de rapidité. Les deux termes $K_c$ et $K_i$ ont tous
les deux une influence sur ces critères : il y a un couplage. Afin de déterminer ces paramètres, une
simulation va être utilisée. Cependant, afin de converger au plus vite, il est nécessaire de trouver un
jeu de valeurs à entrer dans la simulation au plus proche des contraintes du cahier des charges. La
démarche est la suivante :
\begin{itemize}
\item calcul de $K_c$ permettant d’obtenir le temps de réponse à un pour mille (tr$1 \,^0\!/_{00}$) du cahier des
charges;
\item calcul de $K_i$ permettant d’obtenir le dépassement du cahier des charges.
\end{itemize}



\subparagraph{}\textit{Pour cette question, on prend $C_T(p) = K_c$ et $C_r = 0$. Mettre $H(p)$ sous la forme :
$H(p)=\dfrac{K}{1+\tau p}$. Donner les valeurs de $K$ et $\tau$ en fonction de $K_2$, $\tau_2$ et $K_c$. On donne tr$1 \,^0\!/_{00} = 7\tau_1$ pour un
premier ordre. Donner l’expression de la valeur minimale de $K_c$ en fonction de $K_2$, $\tau_2$ et tr$1 \,^0\!/_{00}$ afin
de vérifier le cahier des charges. Calculer alors la valeur de $K_c$ permettant d’avoir un temps de réponse
à $1 \,^0\!/_{00}$ de \SI{135}{s}.}
\ifprof
\begin{corrige}
\end{corrige}
\else
\fi


\subparagraph{}\textit{Pour cette question, on prend $C_T(p)=\dfrac{K_i}{p}$ et $C_r=0$. Mettre $H(p)$ sous la forme $H(p)=\dfrac{K_3}{1+\dfrac{2m}{\omega_0}p+\dfrac{p^2}{\omega_2}}$. Donner les expressions de $K_3$, $m$ et $\omega_0$ en fonction de $K_2$, $\tau_2$ et $K_i$. On donne l'amplitude du premier dépassement (valeur relative) de la réponse indicielle d'un second ordre : $D_1 = \text{exp}\left(-\dfrac{\pi m}{\sqrt{1-m^2}} \right)$. Calculer alors $K_i$ permettant d'avoir un dépassement maximal de $1\%$.}
\ifprof
\begin{corrige}
\end{corrige}
\else
\fi


\subparagraph{}\textit{Les figures suivantes donnent le résultat de la simulation du modèle avec les
valeurs de $K_c$ et $K_i$ trouvées aux questions précédentes. Conclure quant au respect des critères de la
fonction FS5.}
\ifprof
\begin{corrige}
\end{corrige}
\else
\fi




\ifprof
%\end{multicols}
\else
\end{multicols}
\fi

\begin{center}
\includegraphics[width=.8\linewidth]{images/fig_08}
%\textit{Fig 11}
\end{center}

\end{document}

\subparagraph{}\textit{}
\ifprof
\begin{corrige}
\end{corrige}
\else
\fi

\begin{center}
\includegraphics[width=\linewidth]{images/img_04}
%\textit{}
\end{center}

