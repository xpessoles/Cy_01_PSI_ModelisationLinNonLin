\documentclass[10pt,fleqn]{article} % Default font size and left-justified equations
\usepackage[%
    pdftitle={Modélisation SLCI : Rapidité des systèmes},
    pdfauthor={Xavier Pessoles}]{hyperref}
    
\input{style/new_style}
\input{style/macros_SII}
\usepackage{multicol}
\usepackage{siunitx}
%\usepackage{picins}
\fichetrue
%\fichefalse

\proftrue
%\proffalse

\tdtrue
%\tdfalse

\courstrue
\coursfalse

\def\discipline{Sciences \\Industrielles de \\ l'Ingénieur}
\def\xxtete{Sciences Industrielles de l'Ingénieur}

\def\classe{PTSI}
\def\xxnumpartie{Colles}
\def\xxpartie{Modéliser les systèmes asservis dans le but de prévoir leur comportement}


\def\xxnumchapitre{}
\def\xxchapitre{\hspace{.12cm} }


\def\xxtitreexo{Cheville du robot NAO}
\def\xxsourceexo{\hspace{.2cm} \footnotesize{Xavier Pessoles}}


\def\xxposongletx{2}
\def\xxposonglettext{1.45}
\def\xxposonglety{20}
%\def\xxonglet{Part. 1 -- Ch. 3}
\def\xxonglet{Cycle 02}

\def\xxactivite{Colle 1}
\def\xxauteur{\textsl{X. Pessoles}}

\def\xxcompetences{%
\textsl{%
\textbf{Savoirs et compétences :}\\
%Les sources sont associées par un \emph{hacheur série}. La détermination des grandeurs électriques associées à ce montage permet de conclure vis à vis du cahier des charges.
%\noindent \textbf{Résoudre :} à partir des modèles retenus :
%\begin{itemize}[label=\ding{112},font=\color{ocre}] 
%\item choisir une méthode de résolution analytique, graphique, numérique;
%\item mettre en \oe{}uvre une méthode de résolution.
%\end{itemize}
%\begin{itemize}[label=\ding{112},font=\color{ocre}] 
%\item \textit{Rés -- C1.1 :} Loi entrée sortie géométrique et cinématique -- Fermeture géométrique.
%\end{itemize}
%
%\noindent \textit{Mod2 -- C4.1 :} Représentation par schéma bloc.
}}

\def\xxfigures{
\includegraphics[width=.6\linewidth]{images/fig_01}
}%figues de la page de garde


\def\xxpied{%
Modéliser le comportement des systèmes multiphysiques\\
\xxactivite%
}

\setcounter{secnumdepth}{5} 
%---------------------------------------------------------------------------

\usepackage{pgfplots}
\begin{document}

%\chapterimage{png/Fond_Cin}
\input{style/new_pagegarde}
\vspace{5cm}
\pagestyle{fancy}
\thispagestyle{plain}

\def\columnseprulecolor{\color{ocre}}
\setlength{\columnseprule}{0.4pt} 

\def\pathfig{images}

\begin{multicols}{2}
\section*{Présentation}
La figure ci-dessous représente une partie de la chaîne fonctionnelle de la cheville du robot NAO.
\begin{center}
\includegraphics[width=\linewidth]{images/fig_02}
\end{center}

\begin{obj}
Comparer les écarts entre le modèle de la cheville et le comportement réel.
\end{obj}

\subsection*{Modélisation du réducteur}
Le réducteur de la cheville est un réducteur à train simple dont une image CAO est donnée ci-dessous.
\begin{center}
\includegraphics[width=.7\linewidth]{images/fig_03}
\end{center}

Les caractéristiques des roues dentées sont les suivantes : 
\begin{center}
\begin{tabular}{|l|c|c|}
\hline
Pièce & Module & Nombre dents \\ \hline
Pignon 3 20              & 0,3 & 20 \\ \hline
Mobile inf 1 -- Roue    & 0,3 & 80 \\ \hline
Mobile inf 1 -- Pignon  & 0,4 & 25 \\ \hline
Mobile inf 2 -- Roue    & 0,4 & 47 \\ \hline
Mobile inf 2 -- Pignon  & 0,4 & 12 \\ \hline
Mobile inf 4 -- Roue    & 0,4 & 58 \\ \hline
Mobile inf 4 -- Pignon   & 0,7 & 10 \\ \hline
Roue sortie inf          & 0,7 & 36 \\ \hline
\end{tabular}
\end{center}


\subparagraph{}\textit{Donner le rapport de réduction du réducteur $r$.}

\subsection*{Modélisation du moteur à courant continu}
On donne les équations permettant de modéliser le comportement du moteur à courant continu :
\begin{itemize}
\item $u(t) = e(t)+ Ri(t) +L \dfrac{\text{d}i(t)}{\text{d} t}$;
\item $e(t)=K\omega(t)$;
\item $c(t)=Ki(t)$;
\item $c(t)-c_r(t) - f\omega(t)=J\dfrac{\text{d}\omega(t)}{\text{d} t}$.
\end{itemize}

Avec :
\begin{itemize}
\item $u(t)$ : tension d'alimentation du moteur;
\item $i(t)$ : courant circulant dans le moteur;
\item $R$ et $L$ : résistance et inductance du moteur;
\item $K$ : constante électromécanique du moteur;
\item $e(t)$ force contre électromotrice;
\item $\omega(t)$ : taux de rotation du moteur;
\item $J$ : inertie du moteur, du réducteur et de la cheville ramenées à l'arbre moteur;
\item $c(t)$ : couple moteur; 
\item $c_r(t)$ : couple résistant; 
\item $f$ : coefficient de frottement visqueux.
\end{itemize}



\subparagraph{}\textit{Donner les équations dans le domaine de Laplace.}

\vspace{.25cm}

Le moteur à courant continu est commandé par la tension $u(t)$. On mesure le taux de rotation en $\omega(t)$ en sortie. Le système est perturbé par un couple résistant. 

\subparagraph{}\textit{Tracer le schéma-blocs du moteur à courant continu.}

\vspace{.25cm}

On considère que $c_r(t)=0$.

\subparagraph{}\textit{Exprimer la fonction de transfert $\dfrac{\text{d}\omega(t)}{\text{d}u(t)}$.}


\vspace{.25cm}

On considère que $L=\SI{0}{H}$ et $f_v=\SI{0}{Nms}$.

\subparagraph{}\textit{Donner l'expression de la fonction de transfert simplifiée ainsi que le schéma bloc associé.}


Le moteur est sollicité par un échelon de tension $u(t)=U_0 h(t)$ ($h$ fonction de Heaviside).

\subparagraph{}\textit{Quelle est la valeur finale atteinte par $\omega(t)$ ? Quelle est la valeur initiale ? Quelle est la pente à l'origine ?}

\subparagraph{}\textit{Proposer une allure de $\omega(t)$ en fonction du temps.}


\subsection*{Modélisation du système complet}
La cheville est asservie en position angulaire.
\subparagraph{}\textit{Que cela signifie-t-il ?}

\vspace{.25cm}

Le moteur << fournit >> un taux de rotation $\omega(t)$. On souhaite obtenir une angle $\theta(t)$.
\subparagraph{}\textit{Quelle opération mathématique permet de passer d'un taux de rotation à une position angulaire ? Quel est le bloc équivalent  dans le domaine de Laplace ?}

La structure de l'asservissement de la cheville est la suivante :

\begin{center}
\includegraphics[width=\linewidth]{images/fig_04}
\end{center}


\subparagraph{}\textit{Compléter le schéma-blocs.}

\vspace{.25cm}
Lorsque le système est correctement asservi, $\theta_c(p)=\theta(p)$ et $\varepsilon(p)=0$.

\subparagraph{}\textit{Dans ces conditions proposez une technologie de capteur pour le gain $K_{capt}$. Proposer une valeur de gain pour $K_{adapt}$.}


\section*{Conclusion}

\end{multicols}
\end{document}

\subparagraph{}\textit{}


\begin{center}
\includegraphics[width=\linewidth]{images/fig_06}
%\textit{}
\end{center}
\begin{center}
\includegraphics[width=\linewidth]{images/img_04}
%\textit{}
\end{center}


