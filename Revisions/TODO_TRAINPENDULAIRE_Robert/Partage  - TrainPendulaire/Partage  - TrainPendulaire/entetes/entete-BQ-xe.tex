% !TeX encoding = utf8
%%--------------DOCUMENT--------------------------------------------------------

                 % Type de document

\usepackage{etex}

\usepackage{amsmath}
\usepackage{amsfonts}
\usepackage{amssymb}

\usepackage{fontspec}

\usepackage{mathpazo}
%\setmainfont
%     [ BoldFont       = texgyrepagella-bold.otf ,
%       ItalicFont     = texgyrepagella-italic.otf ,
%       BoldItalicFont = texgyrepagella-bolditalic.otf ]
%     {texgyrepagella-regular.otf}

\setmainfont{Times New Roman}

\usepackage[french]{babel} 

%-------------PACKAGES---------------------------------------------------------


\usepackage{geometry}
\geometry{% siehe geometry.pdf (Figure 1)
	bottom=20mm,
	showframe=false, % For debugging: try true and see the layout frames
	margin=20mm,
	marginparsep=3mm,
	marginparwidth=20mm
}


%\usepackage[usenames,dvipsnames]{xcolor}

\usepackage[usenames,dvipsnames,table]{xcolor}
%\usepackage{charter}
\usepackage{xltxtra}

\definecolor{mybluei}{RGB}{28,138,207}
\definecolor{myblueii}{RGB}{131,197,231}

\addtokomafont{disposition}{\selectfont\color{BlueViolet}}

\addtokomafont{chapter}{\fontsize{30pt}{30pt}\selectfont}
\newkomafont{chapternumber}{\fontsize{40}{100}\selectfont\color{black}}
\newkomafont{chaptername}{\itshape\rmfamily\small\color{black}}

\addtokomafont{section}{\fontsize{14pt}{14pt}\selectfont}
\newkomafont{sectionnumber}{\fontsize{18pt}{18pt}\selectfont\rmfamily\color{black}}

\addtokomafont{subsection}{\fontsize{12pt}{12pt}\selectfont}
\newkomafont{subsectionnumber}{\fontsize{16pt}{16pt}\selectfont\rmfamily\color{black}}

\renewcommand\chapterformat{%
  \raisebox{-6pt}{\colorbox{white!70!BlueViolet}{%
    \parbox[b][50pt]{45pt}{\centering%
      {\usekomafont{chaptername}{\chaptername}}%
      \vfill{\usekomafont{chapternumber}{\thechapter\autodot}}%
      \vspace{6pt}%
}}}\enskip}

\renewcommand\sectionformat{%
  \setlength\fboxsep{5pt}%
  \raisebox{-4pt}{\colorbox{white!70!BlueViolet}{%
    \enskip\usekomafont{sectionnumber}{\thesection\autodot}\enskip}%
  \quad%
}}

\renewcommand\subsectionformat{%
  \setlength\fboxsep{5pt}%
  \raisebox{-4pt}{\colorbox{white!70!BlueViolet}{%
    \enskip\usekomafont{subsectionnumber}{\thesubsection\autodot}\enskip}%
  \quad%
}}

\makeatletter
\renewcommand\sectionlinesformat[4]{%
  \makebox[0pt][l]{\rule[-5pt]{\textwidth}{1pt}}%
  \@hangfrom{#3}{#4}%
}
\makeatother

\usepackage{float} 

%\usepackage{fancyhdr}                       % entete et pied de pages
\usepackage[outerbars]{changebar}           % positionnement barre en marge externe
\usepackage{makeidx}                       % Indexation du document


\usepackage{ntheorem}
\theoremstyle{nonumberplain}
\newtheorem{enonce}{Énoncé}

\usepackage{xspace}
\usepackage{relsize}
\usepackage{array} 
\usepackage{mdframed}
\usepackage{etoolbox}


\usepackage{multicol}
\usepackage{textcomp}
\usepackage{subfig}
%\usepackage{epic,bez123}
\usepackage{floatflt}% package for floatingfigure environment
\usepackage{wrapfig}% package for wrapfigure environment
%\usepackage{picins}
\usepackage{placeins}

%\usepackage{csvsimple}
\usepackage[figuresright]{rotating}

\usepackage[french,ruled,linesnumbered,boxed]{algorithm2e}

\usepackage{autoaligne}
\usepackage{paralist}
\usepackage{supertabular}
\usepackage{longtable}

\usepackage{xfrac}


%\usepackage{currfile}

\usepackage{systeme}


%\usepackage{ccicons}
\usepackage{accents}

\usepackage[euler-digits]{eulervm}
\usepackage{siunitx}
\sisetup{output-decimal-marker={,},group-minimum-digits=4,abbreviations, 
%math-micro=\mu,text-micro=\mu, 
 math-ohm=\Omega,
  text-ohm=\ensuremath{\Omega},}

\setcounter{MaxMatrixCols}{20} 


\usepackage{atbegshi} 
\usepackage{tikz}
\usetikzlibrary{fit,chains,matrix,3d,arrows,backgrounds }
\usetikzlibrary{calc,intersections,through,spy}
\usepgflibrary{shapes.geometric,shapes.multipart}
\usepackage{tikz-timing}
\usetikzlibrary{circuits.ee.IEC}
\usepackage[europeanresistors,europeaninductors,europeancurrents,siunitx]{circuitikz}
%\usetikzlibrary{circuits.logic.IEC,circuits.ee.IEC,circuits.logic.US}




%\usepackage{tkz-2d}
\usepackage{tikz-3dplot}


%\usepackage{supertabular}
\usepackage{longtable}

\newcommand{\nlignes}{\Acompleter}

\newcommand{\FeuilleTD}[1][]{\section{Feuille de travaux dirigés n°\thechapter #1}}

\newcommand{\Correction}[1][]{\subsection{Corrigés n°\thechapter#1}}


%-------------PACKAGES PERSO---------------------------------------------------------

\usepackage{packagesRP/schemabloc}
\usepackage{packagesRP/grafcet}
\usepackage{packagesRP/bodegraph}
\usepackage{packagesRP/rpcinematik}
\usepackage{packagesRP/ppnmacro}
\usepackage{packagesRP/rp-sysml}

\usepackage{packagesRP/MyExo}

\usepackage{ifpdf}

  \usepackage{graphicx}

  \DeclareGraphicsExtensions{.png,.pdf,.mps,.eps,.tpx}


\newcommand{\licence}{\includegraphics[height=1.5em]{licence-by-sa.png}}


\newcommand{\NP}{ }
\DeclareMathOperator{\sgn}{sgn}




%-------------ENTETE-ET-PIED-DE-PAGE-------------------------------------------
\usepackage[automark]{scrlayer-scrpage}

\automark[section]{chapter}
\clearpairofpagestyles
\ihead{\headmark}
%\ohead{\pagemark}

%\clearpairofpagestyles
\cfoot[\pagemark]{\pagemark}
%\ihead{\leftmark}
\ohead{}
\ifoot{}
\ofoot{\thepage}
\cfoot{}
\chead{}

\pagestyle{scrheadings}


\setcounter{secnumdepth}{3}
\renewcommand{\thesubsubsection}{\alph{subsubsection}~)}
\makeindex

% divers

\newdimen\oldparindent

%-------------PAGE-DE-GARDE----------------------------------------------------

\title{Manuel de Sciences Industrielles de l'Ingénieur}                                    % Titre

\author{Papanicola Robert\\
Professeur de chaire supérieure \\
au  Lycée Charlemagne-Paris}                                   % Auteur(s)
\date{}     

\usepackage{relsize}

\usepackage{lastpage}
\usepackage{nameref}
\usepackage[francais]{varioref}
\usepackage[plainpages=false,unicode,psdextra]{hyperref}

\usepackage{karnaugh-map}

%------------- commande A5 ---A4 
%

\ifdim\paperheight<150mm%
\let\Acinqclearpage\clearpage
\let \Aquatreclearpage\relax \else%
\let\Acinqclearpage\relax
\let\Aquatreclearpage\clearpage \fi 


%------------------


