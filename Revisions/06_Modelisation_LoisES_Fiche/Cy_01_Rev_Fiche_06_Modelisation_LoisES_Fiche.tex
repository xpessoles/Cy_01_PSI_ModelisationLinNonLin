%%%% Paramétrage du cours %%%%
\def\xxactivite{Cours}
\def\xxauteur{\textsl{Xavier Pessoles}}

\fichetrue
\proftrue
\tdfalse
\coursfalse


\def\xxnumchapitre{Révisions 6 \vspace{.2cm}}
\def\xxchapitre{\hspace{.12cm} Lois Entrée -- Sortie}


\def\xxcompetences{%
\textsl{%
\textbf{Savoirs et compétences :}\\
\begin{itemize}[label=\ding{112},font=\color{ocre}] 
\item **\textit{Mod2.C1 : }Chaîne d’énergie et d'information
%\item \textit{Mod2.C8 : }Linéarisation des systèmes non linéaires	
%\item \textit{Mod3.C1 : }Point de fonctionnement : non-linéarités (hystérésis, saturation, seuil)
\end{itemize}
}}

\def\xxfigures{
%\includegraphics[width=1.4\textwidth]{matlab}%images/prot_01
\\
%\textit{Modèle du pilote hydraulique avec pilotage interactif.}
}%figues de la page de garde


\iflivret
\input{../style/new_pagegarde}
\else
\input{../../style/new_pagegarde}
\fi
\setlength{\columnseprule}{.1pt}

\vspace{1.2cm}
\pagestyle{fancy}
\thispagestyle{plain}



%%%%%%%%%%%%%%%%%%%%%%%
\setcounter{section}{0}
\section{Introduction}
\begin{defi}[Notion de chaîne de solide]
Lorsqu'on modélise un mécanisme par un graphe de structure, on peut distinguer plusieurs types chaînes de solides.


Les lois << entrée -- sortie >> permettent d'établir les relations géométriques ou cinématiques entre les mouvement des actionneurs et les mouvements utiles du mécanisme. 

\textit{Par exemple, pour un moteur thermique, la loi entrée-sortie permet de faire le lien entre la position angulaire du vilebrequin et la position axiale du piston, pour une plate forme hexapode, on peut déterminer la longueur des vérins et la position et l'orientation de la plate-forme \textit{etc.}}
\end{defi}



\begin{defi}
Selon la forme du graphe de liaisons, on peut distinguer 3 cas :
\begin{multicols}{3}
\begin{center}
\textbf{Les chaînes ouvertes} 
\end{center}

\begin{center}
%\includegraphics[width=.6\linewidth]{images/ericc_01}

%\vspace{.5cm}

\includegraphics[width=.5\linewidth]{images/ericc_02}
\end{center}


\vfill\null
\columnbreak

\begin{center}
\textbf{Les chaînes fermées} 
\end{center}

\begin{center}
%\includegraphics[width=.8\linewidth]{images/sympact_01}

%\vspace{.5cm}

\includegraphics[width=.5\linewidth]{images/sympact_02}
\end{center}

\vfill\null
\columnbreak

\begin{center}
\textbf{Les chaînes complexes} 
\end{center}

\begin{center}
%\includegraphics[width=.95\linewidth]{images/haptique_01}

\vspace{-.2cm}

\includegraphics[width=.8\linewidth]{images/haptique_02}
\end{center}

\end{multicols}

On appelle cycle, un chemin fermé ne passant pas deux fois par le même sommet.
À partir d’un graphe des liaisons donné, il est possible de vérifier qu’il existe un nombre
maximal de cycles indépendants. Ce nombre est appelé nombre cyclomatique. 

\textbf{En notant $L$ le nombre de liaisons et $S$ le nombre de solides, on note $\gamma$ le nombre cyclomatique et on a : $\gamma = L - S + 1$.}
\end{defi}


\section{Notions de paramétrage}
Pour réaliser une loi entrée -- sortie il est nécessaire de disposer d'un schéma cinématique paramétré. Cela signifie donc :
\begin{itemize}
\item qu'on associe un repère à chacune des pièces;
\item que chacun de ces repères sont paramétrés les uns par rapport aux autres en définissant les positions et les orientations relatives;
\item que les dimensions internes des pièces sont précisées. 
\end{itemize}

\begin{exemple}[Paramétrage d'une liaison pivot glissant]~\\ Les paramètres $\theta$ et $\lambda$ sont des paramètres variables. $a$ est un paramètre constant. 
\begin{center}
\includegraphics[width=.8\linewidth]{fig_02}
\end{center}
\end{exemple}
\section{Fermeture angulaire}

\noindent\begin{minipage}[c]{.65\linewidth}
\begin{methode}[Fermeture angulaire pour un mouvement plan]~\\
Soient $n$ bases permettant de paramétrer $n$ pièces. Le mouvement est contenu dans le plan $\angl{x_1}{y_1}$. On note $\theta_{i,i+1}=\angl{x_i}{x_{i+1}}=\angl{y_i}{y_{i+1}}$ avec $i \in [1,n-1]$ et $\theta_{1,n}=\angl{x_1}{x_{n}}$. On a alors $\sum\limits_{i=1}^{n-1}\theta_{i,i+1}=\theta_{1,n}$.
\end{methode}
\end{minipage}\hfill
\begin{minipage}[c]{.25\linewidth}
\begin{center}
\includegraphics[width=\linewidth]{fig_03}
$\theta_{12}+\theta_{23}=\theta_{13}$

\end{center}
\end{minipage}

\section{Fermeture géométrique}
\begin{methode}
Réaliser une fermeture géométrique, pour une chaîne fermée, consiste en :
\begin{itemize}
\item écrire une chaîne vectorielle; 
\item la projeter dans la base associée à une des pièces;
\item éliminer les paramètres non souhaités. 
\end{itemize}
Pour écrire la chaîne vectorielle, on utilise souvent le centre des liaisons qui constituent la boucle fermée.
\end{methode}
\section{Fermeture cinématique}

\begin{methode}
Pour réaliser une fermeture de chaîne cinématique d'une chaîne fermée de $n$ pièces et $n$ liaisons, on réalise une fermeture torsorielle : 
$$
\torseurcin{V}{1}{n}= \sum\limits_{i=1}^{n-1}\torseurcin{V}{i}{i+1} \quad \text{soit} \quad 
\torseurl{\vecto{1}{n}}{\vectv{P}{1}{n}}{P}=
\torseurl{\sum\limits_{i=1}^{n-1}\vecto{i}{i+1}}{\sum\limits_{i=1}^{n-1}\vectv{P}{i}{i+1}}{P}
$$ 
\end{methode}


%
%\begin{thebibliography}{2}
%   \bibitem[1]{ref1} Y. Crevits, {\it Éléments de modélisation multi-physique des systèmes industriels en vue de leur simulation numérique, Juin 2015.}
%   \bibitem[2]{ref2} Ph. Fichou, {\it La modélisation multiphysique, Technologie, Mars 2012.}
%   \bibitem[3]{ref3} Yvan Liebgott, {\it Modélisation et Simulation des Systemes Multi-Physiques avec MATLAB.}
%   \bibitem[4]{ref4} Frédéric Mazet, {\it Cours d'automatique de deuxième année, Lycée Dumont Durville, Toulon.}
%   \bibitem[5]{ref5} Patrick Beynet, {\it Sciences industrielles de l'ingénieur MP - PSI}
%, Éditions Ellipses.
%%      \bibitem[5]{ref5} Ivan Liebgott, {\it Modélisation et Simulation des systèmes Multi-Physiques avec MATLAB--Simulink.}
%
%\end{thebibliography}
%
