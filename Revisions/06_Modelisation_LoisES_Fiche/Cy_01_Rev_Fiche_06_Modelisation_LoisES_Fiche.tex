%%%% Paramétrage du cours %%%%
\def\xxactivite{Cours}
\def\xxauteur{\textsl{Xavier Pessoles}}

\fichetrue
\proftrue
\tdfalse
\coursfalse


\def\xxcompetences{%
\textsl{%
\textbf{Savoirs et compétences :}\\
\begin{itemize}[label=\ding{112},font=\color{ocre}] 
\item **\textit{Mod2.C1 : }Chaîne d’énergie et d'information
%\item \textit{Mod2.C8 : }Linéarisation des systèmes non linéaires	
%\item \textit{Mod3.C1 : }Point de fonctionnement : non-linéarités (hystérésis, saturation, seuil)
\end{itemize}
}}

\def\xxfigures{
%\includegraphics[width=1.4\textwidth]{matlab}%images/prot_01
\\
%\textit{Modèle du pilote hydraulique avec pilotage interactif.}
}%figues de la page de garde


\iflivret
\input{../style/new_pagegarde}
\else
\input{../../style/new_pagegarde}
\fi
\setlength{\columnseprule}{.1pt}

\vspace{2cm}
\pagestyle{fancy}
\thispagestyle{plain}



%%%%%%%%%%%%%%%%%%%%%%%
\setcounter{section}{0}
\section{Notions de paramétrage}
\section{Fermeture angulaire}
\section{Fermeture géomtérique}
\section{Fermeture cinématique}


%
%\begin{thebibliography}{2}
%   \bibitem[1]{ref1} Y. Crevits, {\it Éléments de modélisation multi-physique des systèmes industriels en vue de leur simulation numérique, Juin 2015.}
%   \bibitem[2]{ref2} Ph. Fichou, {\it La modélisation multiphysique, Technologie, Mars 2012.}
%   \bibitem[3]{ref3} Yvan Liebgott, {\it Modélisation et Simulation des Systemes Multi-Physiques avec MATLAB.}
%   \bibitem[4]{ref4} Frédéric Mazet, {\it Cours d'automatique de deuxième année, Lycée Dumont Durville, Toulon.}
%   \bibitem[5]{ref5} Patrick Beynet, {\it Sciences industrielles de l'ingénieur MP - PSI}
%, Éditions Ellipses.
%%      \bibitem[5]{ref5} Ivan Liebgott, {\it Modélisation et Simulation des systèmes Multi-Physiques avec MATLAB--Simulink.}
%
%\end{thebibliography}
%
