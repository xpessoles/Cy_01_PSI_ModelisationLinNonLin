\documentclass[10pt,fleqn]{article} % Default font size and left-justified equations
\usepackage[%
    pdftitle={Modélisation systèmes multiphysiques : Diagramme de Bode},
    pdfauthor={Xavier Pessoles}]{hyperref}

\input{style/new_style}
\input{style/macros_SII}

\fichetrue
%\fichefalse

\proftrue
%\proffalse

%\tdtrue
\tdfalse

%\courstrue
\coursfalse

%%%%%%%%%%%%%%%%%%%
\usepackage{numprint}
\usetikzlibrary{calc}
%definition style pour mettre un fond blanc dans un node sans avoir des marges énormes
\tikzset{fondblanc/.style={ inner sep=2pt,fill=white,outer sep = 5pt}} 
\tikzset{fondblanc2/.style={ inner sep=2pt,fill=white}} 
\tikzset{fondblanc3/.style={ inner sep=1pt,fill=white,outer sep = 2pt}} 
\usetikzlibrary{calc,circuits.ee.IEC}
\usetikzlibrary{shapes}
\usepackage[european resistor, european voltage, european current]{circuitikz}
\usetikzlibrary{babel}
\usepackage{standalone}
\standaloneconfig{mode=buildnew}
\usepackage{style/schemabloc}
\usepackage{picins}
\usepackage{style/bodegraph}

% -------------------------------------
% Déclaration des titres
% -------------------------------------

\def\discipline{Sciences \\Industrielles de \\ l'Ingénieur}
\def\xxtete{Sciences Industrielles de l'Ingénieur}

\def\classe{Cy01 - R 03}
\def\xxnumpartie{Cycle 01}
\def\xxpartie{Modéliser le comportement linéaire et non linéaire des systèmes multiphysiques}

\def\xxnumchapitre{Révisions 5 \vspace{.2cm}}
\def\xxchapitre{\hspace{.12cm} Modélisation des systèmes linéaires -- Domaine fréquentiel}

\def\xxposongletx{2}
\def\xxposonglettext{1.45}
\def\xxposonglety{19}%16

\def\xxonglet{Cycle 01 -- Rév 5}

\def\xxactivite{Fiche}
\def\xxauteur{\textsl{Xavier Pessoles}}

\def\xxcompetences{%
\textsl{%
\textbf{Savoirs et compétences :}\\
}}

\def\xxfigures{
%incgraphics[width=.8\textwidth]{}%images/prot_01}
}%figues de la page de garde

\def\xxpied{%
Cycle 01 -- Modéliser le comportement des systèmes multiphysiques\\
Révision 5 -- \xxactivite%
}

\setcounter{secnumdepth}{5}
%---------------------------------------------------------------------------


\begin{document}
%\chapterimage{png/Fond_Cin}
\input{style/new_pagegarde}
\vspace{2cm}
\pagestyle{fancy}
\thispagestyle{plain}

\section{Définitions}

\begin{tikzpicture}[xscale=7/4]
\begin{scope}[yscale=3/40]
\semilog{-2}{2}{-20}{20}
\BodeAmp[cyan,thin,samples=100]{-2:2}
{\POAmpAsymp{6}{0.3}}
\BodeAmp[cyan,thick]{-2:2}{\POAmp{6}{0.3}}
\end{scope}
\begin{scope}[yshift=-2.5cm,yscale=3/90]
\semilog{-2}{2}{-90}{0}
\BodeArg[cyan,samples=100,thin]{-2:2}{\POArgAsymp{6}{0.3}}
\BodeArg[cyan,thick]{-2:2}{\POArg{6}{0.3}}
\end{scope}
\end{tikzpicture}

\section{Gain}
\begin{resultat}[Diagramme de Bode d'un gain pur] ~\\

\noindent\begin{minipage}[c]{.53\linewidth}
\begin{itemize}
\item Fonction de transfert : $H(p)=K$.
\item Diagramme de gain : droite horizontale d'ordonnée $20 \log K$.
\item Diagramme de phase: droite horizontale d'ordonnée 0\degre.
\end{itemize}
\end{minipage} \hfill
\begin{minipage}[c]{.45\linewidth}
\begin{tikzpicture}[xscale=7/5]
\begin{scope}[yscale=3/50]
\semilog{-2}{2}{0}{20}
%\BodeAmp[cyan,thin,samples=100]{-2:2}{\POAmpAsymp{20}{0.2}}
\BodeAmp[cyan,thick]{-2:2}{\KAmp{2}}

\draw (0,8) node [above] {\footnotesize $20 \log K$};
%\draw (1.7,20) node {\footnotesize $-$20 dB/d\'ecade};
%\draw [dashed,black] (.7,0) -- (.7,25);
%\draw (.7,0)  node {\Huge $\cdot$} node [above right]{\footnotesize $\dfrac{1}{\tau}$};
\end{scope}
\begin{scope}[yshift=-1cm,yscale=1/50]
\UniteDegre
\OrdBode{30}
\semilog{-2}{2}{-30}{30}
\draw [thick,cyan] (-2,0) -- (2,0) node [above,midway,black]{\footnotesize 0\degre};
%\BodeArg[cyan,samples=100,thin]{-2:2}{\POArgAsymp{20}{0.2}}
%\BodeArg[cyan,thick]{-2:2}{\POArg{20}{0.2}}
\end{scope}
\end{tikzpicture}

\end{minipage}
\end{resultat}
\section{Intégrateur}
\section{Dérivateur}
\section{Systèmes d'ordre 1}
\begin{resultat}[Diagramme de Bode d'un système du premier ordre] ~\\

\noindent\begin{minipage}[c]{.53\linewidth}
\begin{itemize}
\item Fonction de transfert : $H(p)=\dfrac{K}{1+\tau p}$.
\item Diagramme de gain asymptotique : 
\begin{itemize}
\item pour $\omega<\dfrac{1}{\tau}$ : droite horizontale d'ordonnée $20 \log K$;
\item pour $\omega>\dfrac{1}{\tau}$ : droite de pente $-{20}\text{dB/decade}$.
\end{itemize}
\item Diagramme de phase asymptotique : 
\begin{itemize}
\item pour $\omega<\dfrac{1}{\tau}$ : droite horizontale d'ordonnée 0 \degre;
\item pour $\omega>\dfrac{1}{\tau}$ : droite horizontale d'ordonnée $-90$ \degre.
\end{itemize}
\end{itemize}
\end{minipage} \hfill
\begin{minipage}[c]{.45\linewidth}
\begin{tikzpicture}[xscale=7/5]
\begin{scope}[yscale=3/50]
\semilog{-2}{2}{0}{30}
\BodeAmp[cyan,thin,samples=100]{-2:2}{\POAmpAsymp{20}{0.2}}
\BodeAmp[cyan,thick]{-2:2}{\POAmp{20}{0.2}}
\draw (-1,28) node {\footnotesize $20\log K$, 0 dB/d\'ecade};
\draw (1.7,20) node {\footnotesize $-$20 dB/d\'ecade};
\draw [dashed,ultra thick,black] (.7,0) -- (.7,25);
\draw (.7,0)  node {\Huge $\cdot$} node [above right]{\footnotesize $\dfrac{1}{\tau}$};
\end{scope}
\begin{scope}[yshift=-0.5cm,yscale=1/50]
\UniteDegre
\OrdBode{30}
\semilog{-2}{2}{-90}{0}
\BodeArg[cyan,samples=100,thin]{-2:2}{\POArgAsymp{20}{0.2}}
\BodeArg[cyan,thick]{-2:2}{\POArg{20}{0.2}}
\end{scope}
\end{tikzpicture}

\end{minipage}
\end{resultat}

\section{Systèmes d'ordre 2}
%
%\begin{defi}
%
%Les systèmes du premier ordre sont régis par une équation différentielle de la
%forme suivante :
%$$
%\tau \dfrac{\dd s(t)}{\dd t}+s(t) = Ke(t).
%$$
%
%\begin{minipage}[c]{.6\linewidth}
%Dans le domaine de Laplace, la fonction de transfert de ce système est donc
%donnée par :
%
%$$
%H(p)=\dfrac{S(p)}{E(p)} = \dfrac{K}{1+\tau p}
%$$
%
%On note :
%\begin{itemize}
% \item $\tau$ la constante de temps en secondes ($\tau>0$);
%\item $K$ le gain statique du système ($K>0$).
%\end{itemize}
%\end{minipage}\hfill
%\begin{minipage}[c]{.35\linewidth}
%
%%Schéma-bloc d'un système du premier ordre :
%
%\begin{center}
%\begin{tikzpicture}
%\sbEntree{E}
%\sbBloc[5]{B1}{$H(p)=\dfrac{K}{1+\tau p}$}{E}
%\sbSortie[5]{S}{B1}
%\sbRelier[$E(p)$]{E}{B1}
%\sbRelier[$S(p)$]{B1}{S}
%\end{tikzpicture}
%\end{center}
%\end{minipage}
%\end{defi}
%
%
%
%
%\begin{resultat}[Réponse à un échelon d'un système du premier ordre]~\\
%On appelle réponse à un échelon, l'expression de la sortie $s$ lorsque on soumet le système à un échelon d'amplitude $E_0$. Lorsque $E_0=1$ ($1/p$ dans le domaine de Laplace) on parle de \textbf{réponse indicielle}.
%Ainsi, dans le domaine de Laplace :
%$$
%S(p)=E(p)H(p) = \dfrac{E_0}{p} \dfrac{K}{1+\tau p}.
%$$ 
%
%Analytiquement, on montre que $s(t)=K E_0 u(t) \left(1-e^{-\frac{t}{\tau}}\right)$. 
%
%
%\noindent \begin{minipage}[c]{.65\linewidth}
%
%Si la réponse indicielle d'un système est caractéristique d'un modèle du premier ordre (pente à l'origine non nulle et pas d'oscillation), on détermine :
%\begin{itemize}
%\item le gain à partir de l'asymptote $K E_0$;
%\item la constante de temps à partir de $t_{5\%}$ ou du temps pour $63~\%$ de la valeur finale.% (ou $3\tau$ pour $95~\%$ de la valeur finale).
%\end{itemize}
%Les caractéristiques de la courbe sont les suivantes : 
%\begin{itemize}
%\item valeur finale $s_{\infty}=K E_0$;
%\item pente à l'origine \textbf{non nulle};
%\item $t_{5\%}=3\tau$;
%\item pour $t=\tau$, $s(\tau)=0,63~ s_{\infty}$.
%%\item Plus $\tau$ est grand, plus le système est lent.
%\end{itemize}
%\end{minipage} \hfill
%\begin{minipage}[c]{.32\linewidth}
%\centering
%\includestandalone{images/fig_01}
%\end{minipage}
%\end{resultat}
%
%
%\begin{resultat}[Réponse à un échelon d'un système du deuxième ordre]~\\
%
%\noindent \begin{minipage}[c]{.65\linewidth}
%On appelle réponse à une rampe, l'expression de la sortie $s$ lorsque on soumet le système à une fonction linéaire de pente $k$: 
%$$
%S(p)=E(p)H(p) = \dfrac{k}{p^2} \dfrac{K}{1+\tau p}.
%$$ 
%
%
%Analytiquement, on montre que $s(t)=Kk \left(t-\tau+\tau e^{-\frac{t}{\tau}}\right)u(t)$. 
%
%Les caractéristiques de la courbe sont les suivantes : 
%\begin{itemize}
%\item pente de l'asymptote $K k$;
%%\item pente à l'origine \textbf{non nulle};
%\item intersection de l'asymptote avec l'axe des abscisses : $t=\tau$;
%\item $\varepsilon_{v}=kK\tau$.
%%\item Plus $\tau$ est grand, plus le système est lent.
%\end{itemize}
%
%
%\end{minipage} \hfill
%\begin{minipage}[c]{.32\linewidth}
%\centering
%\includestandalone{images/rampe}
%\end{minipage}
%\end{resultat}
%
%
%
%\section{Systèmes d'ordre 2}
%
%\begin{defi}
%
%Les systèmes du premier ordre sont régis par une équation différentielle de la
%forme suivante :
%$$
%\dfrac{1}{\omega_0^2} \dfrac{\dd^2 s(t)}{\dd t^2}+\dfrac{2\xi}{\omega_0} \dfrac{\dd s(t)}{\dd t}+s(t) = Ke(t).
%$$
%
%Dans le domaine de Laplace, la fonction de transfert de ce système est donc
%donnée par :
%
%\begin{minipage}[c]{.6\linewidth}
%$$
%H(p)=\dfrac{S(p)}{E(p)} = \dfrac{K}{1+ \dfrac{2\xi}{\omega_0}p+\dfrac{p^2}{\omega_0^2}}.
%$$
%\end{minipage}\hfill
%\begin{minipage}[c]{.35\linewidth}
%%Schéma-bloc d'un système du second ordre :
%
%\begin{center}
%\begin{tikzpicture}
%\sbEntree{E}
%\sbBloc[5]{B1}{$\dfrac{K}{1+ \dfrac{2\xi}{\omega_0}p+\dfrac{p^2}{\omega_0^2}}$}{E}
%\sbSortie[5]{S}{B1}
%\sbRelier[$E(p)$]{E}{B1}
%\sbRelier[$S(p)$]{B1}{S}
%\end{tikzpicture}
%\end{center}
%\end{minipage}
%
%
%On note :
%\begin{itemize}
%\item $K$ est appelé le gain statique du système (rapport des unités de $S$ et de $E$);
%\item $\xi$ (lire \textit{xi}) est appelé coefficient d'amortissement (sans unité);
%\item $\omega_0$ pulsation propre du système ($\text{rad/s}$ ou $s^{-1}$).
%\end{itemize}
%
%
%Suivant la valeur du coefficient d'amortissement, l'allure de la réponse temporelle est différente.
%\end{defi}
%
%\begin{resultat} ~\\
%
%\vspace*{-1.3cm}
%
%\noindent\begin{center}
%\begin{tabular}{p{.4\linewidth}p{.58\linewidth}}
%\begin{center}
%\textbf{$z\geq 1$ : système non oscillant et amorti}
%
%\textbf{(apériodique)}
%
%\includestandalone{images/Ordre2_amorti}
%\end{center} 
%& 
%\begin{center}
%\textbf{$z<1$ : système oscillant et amorti }
%
%\textbf{(pseudo périodique)}
%
%\includestandalone{images/Ordre2_pseudo}
%\end{center} 
%\\
%\vspace*{-1cm}
%\begin{itemize} 
%\item La fonction de transfert a deux pôles réels.
%\item La tangente à l'origine est nulle.
%\end{itemize}
%& 
%\vspace*{-.8cm}
%\begin{itemize} 
%\item La fonction de transfert a deux pôles réels.
%\item La tangente à l'origine est nulle.
%\item La pseudo-période est de la forme $T_p=\dfrac{2\pi }{\omega_0 \sqrt{1-\xi^2}}$.
%\item La valeur du premier dépassement vaut : $D_1=e^{\dfrac{-\pi \xi }{\sqrt{1-\xi^2}}}$.
%\end{itemize}
%\end{tabular}
%\end{center}
%\end{resultat}
%
%\begin{resultat} ~\\
%\begin{itemize}
%\item Pour $\xi\simeq 0,7$ le système du second ordre le temps à un de réponse à 5\% le plus petit \textbf{avec dépassement} et $t_{r 5\%} \cdot \omega_0 \simeq 3$.
%\item Pour $\xi =1$ on obtient le système du second ordre plus rapide \textbf{sans dépassement}.
%
%\end{itemize}
%\end{resultat}

\end{document}


