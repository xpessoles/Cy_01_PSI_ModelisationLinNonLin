\documentclass[10pt,fleqn]{article} % Default font size and left-justified equations
\usepackage[%
    pdftitle={Modélisation systèmes multiphysiques : Modélisation linéaire et non linéaire},
    pdfauthor={Xavier Pessoles}]{hyperref}
    
\input{style/new_style}
\input{style/macros_SII}
\usepackage{multicol}
\usepackage{siunitx}
\fichetrue
%\fichefalse

\proftrue
%\proffalse

\tdtrue
%\tdfalse

\courstrue
\coursfalse

\def\discipline{Sciences \\Industrielles de \\ l'Ingénieur}
\def\xxtete{Sciences Industrielles de l'Ingénieur}

\def\classe{PSI$\star$ -- MP}
\def\xxnumpartie{\textsf{Cycle 01}}
\def\xxpartie{Modéliser le comportement linéaire et non linéaire des systèmes multiphysiques}


\def\xxnumchapitre{Chapitre 1 \vspace{.2cm}}
\def\xxchapitre{\hspace{.12cm} Modélisation multiphysique}


\def\xxtitreexo{QCM}%Motorisation du moteur Haibike}
\def\xxsourceexo{\hspace{.2cm} \footnotesize{X. Pessoles}}%P. Beynet, Éd Ellipses.}}


\def\xxposongletx{2}
\def\xxposonglettext{1.45}
\def\xxposonglety{20}
%\def\xxonglet{Part. 1 -- Ch. 3}
\def\xxonglet{\textsf{Cycle 01}}

\def\xxactivite{QCM 02}
\def\xxauteur{\textsl{X. Pessoles}}

\def\xxcompetences{%
\textsl{%
\textbf{Savoirs et compétences :}\\
%Les sources sont associées par un \emph{hacheur série}. La détermination des grandeurs électriques associées à ce montage permet de conclure vis à vis du cahier des charges.
%\noindent \textbf{Résoudre :} à partir des modèles retenus :
%\begin{itemize}[label=\ding{112},font=\color{ocre}] 
%\item choisir une méthode de résolution analytique, graphique, numérique;
%\item mettre en \oe{}uvre une méthode de résolution.
%\end{itemize}
%\begin{itemize}[label=\ding{112},font=\color{ocre}] 
%\item \textit{Rés -- C1.1 :} Loi entrée sortie géométrique et cinématique -- Fermeture géométrique.
%\end{itemize}
%
%\noindent \textit{Mod2 -- C4.1 :} Représentation par schéma bloc.
}}

\def\xxfigures{
%\includegraphics[width=.9\linewidth]{images/c-evolution}
}%figues de la page de garde


\def\xxpied{%
Cycle 01 -- Modéliser le comportement des systèmes multiphysiques\\
Chapitre 1 -- \xxactivite%
}

\setcounter{secnumdepth}{5}
%---------------------------------------------------------------------------

\usepackage{pgfplots}
\begin{document}
\def\pathfig{images}
%\chapterimage{png/Fond_Cin}
\input{style/new_pagegarde}
\vspace{6cm}
\pagestyle{fancy}
\thispagestyle{plain}

\def\columnseprulecolor{\color{ocre}}
\setlength{\columnseprule}{0.4pt} 

\def\pathfig{images}

\begin{multicols}{2}
%\end{multicols}


\subparagraph{}\textit{L'exosquelette qui faisait l'objet de l'exercice du jour était :}
\begin{enumerate}
\item est un système suiveur
\item est un système régulateur
\item n'est pas vraiment un système asservi
\item la réponse D
\item je suis désolé Monsieur, j'ai pas fait l'exo, j'ai fait un exo de physique.
\end{enumerate}

\subparagraph{}\textit{On s'intéresse à l'asservissement en pression dans le circuit hydraulique d'un aquarium de maison.}
\begin{enumerate}
\item le système est régulateur.
\item le système est suiveur.
\item non seulement je n'ai pas fais mon exo, mais j'ai pas lu le cours.
\item le système est inutile.
\end{enumerate}

\subparagraph{}\textit{L'écart statique se mesure à partir d'une consigne}
\begin{enumerate}
\item échelon
\item rampe
\item parabolique
\item sinusoïdale
\end{enumerate}

\subparagraph{}\textit{L'écart dynamique se mesure à partir d'une consigne}
\begin{enumerate}
\item échelon
\item rampe
\item parabolique
\item sinusoïdale
\end{enumerate}

\subparagraph{}\textit{L'écart de trainage se mesure à partir d'une consigne}
\begin{enumerate}
\item échelon
\item rampe
\item parabolique
\item sinusoïdale
\end{enumerate}

\subparagraph{}\textit{L'écart en vitesse se mesure à partir d'une consigne}
\begin{enumerate}
\item échelon
\item rampe
\item parabolique
\item sinusoïdale
\end{enumerate}

\subparagraph{}\textit{On donne la courbe suivante.}
\begin{enumerate}
\item L'écart statique est nul.
\item L'écart statique est de 0,2.
\item L'écart statique est de 0,8.
\item L'écart statique est de 20.
\item L'écart statique est infini.
\end{enumerate}

\subparagraph{}\textit{On donne la courbe suivante.}
\begin{center}
\includegraphics[width=6cm]{images/Ord2_Ech_K}
\end{center}
\begin{enumerate}
\item Le temps de réponse est infini.
\item Le temps de réponse est nul.
\item Le temps de réponse est inférieur à 1s.
\item Le temps de réponse est supérieur à 1.
\item En général (en SII), le temps de réponse est mesuré à 5\% de la
consigne.
\end{enumerate}

\subparagraph{}\textit{Dans les conditions de Heaviside, la transformée de Laplace de df(t)/dt est}
\begin{enumerate}
\item pF(p)
\item p²F(p)
\item Les conditions de Heavi qui ?
\item F(p)
\item p
\end{enumerate}

\subparagraph{}\textit{Le théorème de la valeur initiale est donné par :}
\begin{enumerate}
\item la limite de f(t) quand t tend vers 0 est égale à la limite de pF(p) quand p tend vers l'infini.
\item la limite de f(t) quand t tend vers 0 est égale à la limite de F(p) quand p tend vers l'infini.
\item la limite de f(t) quand t tend vers 0 est égale à la limite de pF(p) quand p tend vers 0.
\item la limite de f(t) quand t tend vers l'infini est égale à la limite de pF(p) quand p tend vers 0.
\item la limite de f(t) quand t tend vers 0 est égale à la limite de F(p) quand p tend vers 0.
\end{enumerate}

\subparagraph{}\textit{Le théorème du retard est donné par :}
\begin{enumerate}
\item L[f(t-t0)] = exp(-t0*p) * F(p)
\item L[f(t-t0)] = exp(-t0*p) * F(p+a)
\item L[f(t)] = exp(-t0*p) * F(p)
\item L[f(t0)] = exp(-t0*p) * F(p)
\item L[f(t0)] = exp(-t0*p) * F(p+a)
\end{enumerate}

\subparagraph{}\textit{On donne 
$
H(p) = \dfrac{N(p)}{D(p)} =
K \dfrac{\left(p-z_1 \right)\left(p-z_2 \right)...\left(p-z_m \right)}{
p^{\alpha} \left(p-p_1 \right)\left(p-p_2 \right)...\left(p-p_n \right)}
$. 
Cocher les propositions justes.}
\begin{enumerate}
\item Les zi sont les zéros de la fonction de transfert (réels ou complexes).
\item Les pi sont les pôles de la fonction de transfert (réels ou complexes).
\item Le degré de D(p) est appelé ordre n du système.
\item L’équation D(p) = 0 est appelée équation caractéristique.
\item Le facteur constant K est appelé gain du système.
\end{enumerate}

\end{multicols}
\end{document}