%%%% Paramétrage du TD %%%%
\def\xxactivite{Application 02 \ifprof -- Corrigé \else \fi }
\def\xxauteur{\textsl{Xavier Pessoles}}


\def\xxnumchapitre{Révisions 3 \vspace{.2cm}}
\def\xxchapitre{\hspace{.12cm} Modélisation par fonction de transfert et schéma-blocs}

\def\xxcompetences{%
\textsl{%
\textbf{Savoirs et compétences :}\\
\vspace{-.4cm}
\begin{itemize}[label=\ding{112},font=\color{ocre}] 
%\item \textit{Mod3.C2 : } pôles dominants et réduction de l’ordre du modèle : principe, justification
%\item \textit{Res2.C4 : } stabilité des SLCI : définition entrée bornée -- sortie bornée (EB -- SB)	
%\item \textit{Res2.C5 : } stabilité des SLCI : équation caractéristique	
%\item \textit{Res2.C6 : } stabilité des SLCI : position des pôles dans le plan complexe
\item ...%\textit{Res2.C7 : } stabilité des SLCI : marges de stabilité (de gain et de phase)
\end{itemize}
}}


\def\xxfigures{
%\includegraphics[width=.9\linewidth]{fig_00}
}%figures de la page de garde

\def\xxtitreexo{Application}
\def\xxsourceexo{}

\iflivret
\input{../../style/new_pagegarde}
\else
\input{../../style/new_pagegarde}
\fi
\setlength{\columnseprule}{.1pt}

\pagestyle{fancy}
\thispagestyle{plain}


\vspace{4.5cm}

\def\columnseprulecolor{\color{ocre}}
\setlength{\columnseprule}{0.4pt} 

%%%%%%%%%%%%%%%%%%%%%%%


\ifprof
\else
\begin{multicols}{2}
\fi
\section*{Modélisation par schéma-blocs}

\begin{methode}
Dans le cas où vous ne savez pas comment démarrer, vous pouvez suivre la méthode suivante.
\begin{enumerate}
\item Identifier la grandeur physique d'entrée et la grandeur physique de sortie.
\item Lorsqu'une équation lie deux grandeurs physiques, réaliser le schéma-blocs associé à l'équation. 
\item Lorsqu'une équation lie trois grandeurs physiques, réaliser le schéma-blocs associé à l'équation en utilisant un comparateur.
\item Relier les blocs en commençant par l'entrée. Inverser les blocs si nécessaire.
\end{enumerate}
\end{methode}
\subsection*{Modélisation du moteur à courant continu}

\ifprof
\else
On donne les équations du moteur à courant continu :
\begin{itemize}
\item $u(t) = e(t)+ Ri(t) +L \dfrac{\text{d}i(t)}{\text{d} t}$;
\item $e(t)=K\omega(t)$;
\item $c(t)=Ki(t)$;
\item $c(t)-c_r(t) - f\omega(t)=J\dfrac{\text{d}\omega(t)}{\text{d} t}$.
\end{itemize}
\fi

\subparagraph{}
\textit{Réaliser le schéma-blocs.}

\ifprof
\begin{corrige}~\\
\begin{center}
\includegraphics[width=.7\linewidth]{cor_01}
\end{center}
\end{corrige}
\else
\fi

\subparagraph{}
\textit{Exprimer $\Omega(p)$ sous la forme  $\Omega(p)=F_1(p)U(p) + F_2(p) C_r(p)$. Les fonctions de transfert $F_1$ $F_2$ seront exprimées sous forme canonique. Les constantes du système du second ordre seront explicitées.}
\ifprof
\begin{corrige}
Par superposition, on a : 
$\Omega_1(p)/U(p)=\dfrac{K\dfrac{1}{R+Lp}\dfrac{1}{Jp+f}}{1+K^2\dfrac{1}{R+Lp}\dfrac{1}{Jp+f}}$
$=\dfrac{K}{\left(Jp+f\right)\left(Lp+R\right)+K^2}$.

Par ailleurs, $\Omega_2(p)/C_r(p)=\dfrac{\dfrac{1}{Jp+f}}{1+K^2\dfrac{1}{R+Lp}\dfrac{1}{Jp+f}}$
$=\dfrac{Lp+R}{\left(Jp+f\right)\left(Lp+R\right)+K^2}$.


Au final, $\Omega(p)=\dfrac{K}{\left(Jp+f\right)\left(Lp+R\right)+K^2} U(p) + \dfrac{Lp+R}{\left(Jp+f\right)\left(Lp+R\right)+K^2} C_r(p)$. 

On peut alors mettre $F_1$ sous forme canonique : 
$$
K_0 =\dfrac{K}{fR+K^2}
\quad 
\dfrac{2\xi}{\omega_0} = \dfrac{RJ+Lf}{fR+K^2}
\quad
\dfrac{1}{\omega_0^2} =  \dfrac{LJ}{fR+K^2}.
$$
\end{corrige}
\else
\fi


\subsection*{Modélisation d'une servo-commande}

\setcounter{exo}{0}

\ifprof
\else

\begin{center}
\includegraphics[width=.9\linewidth]{img5}
\end{center}

Les différentes équations temporelles qui modélisent le fonctionnement d'une servocommande sont :
\begin{itemize}
\item un amplificateur différentiel défini par : $u_c(t)=\dfrac{i(t)}{K_a}+u_s(t)$;
\item débit dans le vérin dans le cas d'une hypothèse de fluide incompressible $q(t)=S\cdot\dfrac{dx(t)}{dt}$;
\item capteur de position : $u_s(t)=K_c\cdot x(t)$;
\item le servo-distributeur est un composant de la chaîne de commande conçu pour fournir un débit hydraulique $q(t)$ proportionnel au courant de commande $i(t)$. (Attention, valable uniquement en régime permanent.) Le constructeur fournit sa fonction de transfert :
$$
F(p)=\dfrac{Q(p)}{I(p)}=\dfrac{K_d}{1+Tp}
$$
où $K_d$ est le gain du servo-distributeur et $T$ sa constante de temps.
\end{itemize}
\fi

\subparagraph{}
\textit{Réaliser le schéma-blocs.}
\ifprof
\begin{corrige}
On a :
\begin{itemize}
\item $U_c(p)=\dfrac{1}{K_a}I(p)+U_s(p)$
\item $Q(p)=SpX(p)$
\item $U_S(p)=K_C\cdot X(p)$
\item $F(p)=\dfrac{Q(p)}{I(p)}=\dfrac{K_d}{1+Tp}$
\end{itemize}


\begin{minipage}[c]{.23\linewidth}
\begin{center}
\includegraphics[width=.95\textwidth]{bloc1}
\end{center}
\end{minipage}\hfill
\begin{minipage}[c]{.23\linewidth}
\begin{center}
\includegraphics[width=.95\textwidth]{bloc2}
\end{center}
\end{minipage}\hfill
\begin{minipage}[c]{.23\linewidth}
\begin{center}
\includegraphics[width=.95\textwidth]{bloc3}
\end{center}
\end{minipage}\hfill
\begin{minipage}[c]{.23\linewidth}
\begin{center}
\includegraphics[width=.95\textwidth]{bloc4}
\end{center}
\end{minipage}



\begin{center}
\includegraphics[width=.8\textwidth]{schema_bloc}
\end{center}
\end{corrige}
\else 
\fi


\subparagraph{}
\textit{Déterminer la fonction de transfert en boucle fermée.}





\section*{Réduction de schéma-blocs}
\textit{D'après ressources de V. Reydellet.}
\subparagraph*{}
\textit{Réduire les schéma-blocs suivants.}
\begin{center}
\includegraphics[scale=.2]{sb_01}
\end{center}

\ifprof
\begin{corrige}
~\\
\begin{itemize}
\item Boucle intérieure : $\dfrac{H_1H_2}{1+H_1H_2H_3}$.
\item $H(p)=\dfrac{S(p)}{E(p)}=H_4(p)+\dfrac{H_1(p)H_2(p)}{1+H_1(p)H_2(p)H_3(p)}$.
\end{itemize}
\end{corrige}
\else
\fi

\begin{center}
\includegraphics[scale=.2]{sb_02}
\end{center}

\ifprof
\begin{corrige}
~\\
\begin{itemize}
\item On décale le point de prélèvement de droite vers la gauche (ce qui nécessite alors d'ajouter un bloc $C$ dans la boucle de retour supérieure). La boucle intérieure peut donc être mise sous la forme : $\dfrac{B}{1+BCE}$.
\item Réduction de la boucle totale : $C \dfrac{A\dfrac{B}{1+BCE}}{1+AD\dfrac{B}{1+BCE}}$ $ =\dfrac{ABC}{1+BCE+ABD}$.
\end{itemize}
\end{corrige}
\else
\fi


\section*{Transformation de schéma-blocs}
\subparagraph*{}
\textit{Transformer le schéma-bloc suivant pour obtenir un schéma-blocs à retour unitaire.}
\begin{center}
\includegraphics[scale=.2]{sb_03}
\end{center}



\ifprof
\begin{corrige}
~\\
D'une part, on a $\dfrac{S(p)}{E(p)}=\dfrac{H_1(p)}{1+H_1(p)H_2(p)}$.

D'autre part, on prend un bloc $X_1(p)$ en série avec $X_2(p)$ << possédant >> un retour unitaire. On a donc 
$\dfrac{S(p)}{E(p)}=X_1(p)\dfrac{X_2(p)}{1+X_2(p)}$. 

On a donc $H_1=X_1X_2$ et $X_2=H_1H_2$; donc $X_1  = \dfrac{H_1}{X_2}=\dfrac{H_1}{H_1H_2}=\dfrac{1}{H_2}$.
\end{corrige}
\else
\fi


\subparagraph*{}
\textit{Modifier le schéma-blocs suivant pour obtenir la forme proposée. Déterminer ensuite l'expression de $S(p)$ en fonction de $E(p)$ et $P(p)$.}
\begin{center}
\includegraphics[scale=.2]{sb_04}
\end{center}

\begin{center}
\includegraphics[scale=.2]{sb_05}
\end{center}



%
%\ifprof
%\begin{corrige}
%~\\
%D'une part, on a $S(p)=\left(\varepsilon(p)A(p)+P(p)\right)B(p)C(p)$ et
%$\varepsilon(p)=E(p)-\left(\varepsilon(p)A(p)+P(p)\right)B(p)$
%$\Leftrightarrow \varepsilon(p)=E(p)-\varepsilon(p)A(p)B(p)+P(p)B(p)$
%$\Leftrightarrow \varepsilon(p)\left(1+A(p)B(p)\right)=E(p)+P(p)B(p)$.
%
%Au final, $\varepsilon(p)=\dfrac{1}{1+A(p)B(p)}E(p)+P(p)\dfrac{B(p)}{1+A(p)B(p)}$ et 
% $S(p)=\left(\left(\dfrac{E(p)}{1+A(p)B(p)}+\dfrac{P(p)B(p)}{1+A(p)B(p)}\right)A(p)+P(p)\right)B(p)C(p)$
% 
%  $S(p)=\left(\left(\dfrac{A(p)E(p)}{1+A(p)B(p)}+\dfrac{P(p)A(p)B(p)}{1+A(p)B(p)}\right)+P(p)\right)B(p)C(p)$
%\end{corrige}
%\else
%\fi


\ifprof
\else
 \end{multicols}
\fi
