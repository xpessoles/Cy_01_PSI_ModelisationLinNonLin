\documentclass[10pt,fleqn]{article} % Default font size and left-justified equations
\usepackage[%
    pdftitle={Modélisation systèmes multiphysiques : Modélisation par fonction de transfert et schéma-blocs},
    pdfauthor={Xavier Pessoles}]{hyperref}

\input{style/new_style}
\input{style/macros_SII}

\fichetrue
%\fichefalse

\proftrue
%\proffalse

%\tdtrue
\tdfalse

%\courstrue
\coursfalse

%%%%%%%%%%%%%%%%%%%
\usepackage{numprint}
\usetikzlibrary{calc}
%definition style pour mettre un fond blanc dans un node sans avoir des marges énormes
\tikzset{fondblanc/.style={ inner sep=2pt,fill=white,outer sep = 5pt}} 
\tikzset{fondblanc2/.style={ inner sep=2pt,fill=white}} 
\tikzset{fondblanc3/.style={ inner sep=1pt,fill=white,outer sep = 2pt}} 
\usetikzlibrary{calc,circuits.ee.IEC}
\usetikzlibrary{shapes}
\usepackage[european resistor, european voltage, european current]{circuitikz}
\usetikzlibrary{babel}
\usepackage{standalone}
\standaloneconfig{mode=buildnew}

\usepackage{picins}

% -------------------------------------
% Déclaration des titres
% -------------------------------------

\def\discipline{Sciences \\Industrielles de \\ l'Ingénieur}
\def\xxtete{Sciences Industrielles de l'Ingénieur}

\def\classe{Cy01 - R 03}
\def\xxnumpartie{Cycle 01}
\def\xxpartie{Modéliser les systèmes asservis -- Modélisation par fonction de transfert et schéma-blocs}

\def\xxnumchapitre{Révisions 3 \vspace{.2cm}}
\def\xxchapitre{\hspace{.12cm} Modéliser les systèmes asservis -- Modélisation par fonction de transfert et schéma-blocs}

\def\xxposongletx{2}
\def\xxposonglettext{1.45}
\def\xxposonglety{19}%16

\def\xxonglet{Cycle 01 -- Rév 3}

\def\xxactivite{Fiche}
\def\xxauteur{\textsl{Xavier Pessoles}}

\def\xxcompetences{%
\textsl{%
\textbf{Savoirs et compétences :}\\
}}

\def\xxfigures{
%incgraphics[width=.8\textwidth]{}%images/prot_01}
}%figues de la page de garde

\def\xxpied{%
Cycle 01 -- Modélisation des systèmes linéaires et non linéaires\\
Révision 3 -- \xxactivite%
}

\setcounter{secnumdepth}{5}
%---------------------------------------------------------------------------


\begin{document}
%\chapterimage{png/Fond_Cin}
\input{style/new_pagegarde}
\vspace{2cm}
\pagestyle{fancy}
\thispagestyle{plain}

\section{Définitions}

%\begin{itemize}[label=\ding{112},font=\color{ocre}] 
%\item Une
%\item Une
%\end{itemize}

\begin{defi}[Fonction de transfert] ~\\


\end{defi}

\begin{defi}[Classe et ordre] ~\\


\end{defi}


\begin{defi}[Pôles et Zéros] ~\\


\end{defi}


\begin{defi}[Modélisation d'un bloc] ~\\

\end{defi}

\begin{defi}[Modélisation d'un comparateur] ~\\

\end{defi}


\section{Algèbre de blocs}


\begin{warn}~\\

\end{warn}

\begin{resultat}[Blocs en série] ~\\

\end{resultat}


\begin{resultat}[Blocs en parallèle] ~\\

\end{resultat}


\begin{resultat}[Réduction de boucle] ~\\

\end{resultat}


\begin{resultat}[Comparateurs en série] ~\\

\end{resultat}


\begin{resultat}[Point de prélèvement] ~\\

\end{resultat}



\section{Fonctions usuelles}


\begin{defi}[Fonction de transfert en boucle fermée -- FTBF] ~\\

\end{defi}



\begin{defi}[Fonction de transfert en boucle ouverte -- FTBO] ~\\

\end{defi}


\begin{defi}[Chaîne directe] ~\\

\end{defi}



\begin{defi}[L'écart] ~\\

\end{defi}


\begin{defi}[Théorème de superposition] ~\\

\end{defi}





\end{document}


