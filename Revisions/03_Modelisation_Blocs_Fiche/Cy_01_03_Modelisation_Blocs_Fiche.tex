\documentclass[10pt,fleqn]{article} % Default font size and left-justified equations
\usepackage[%
    pdftitle={Modélisation systèmes multiphysiques : Modélisation de Laplace},
    pdfauthor={Xavier Pessoles}]{hyperref}

\input{style/new_style}
\input{style/macros_SII}

\fichetrue
%\fichefalse

\proftrue
%\proffalse

%\tdtrue
\tdfalse

%\courstrue
\coursfalse

%%%%%%%%%%%%%%%%%%%
\usepackage{numprint}
\usetikzlibrary{calc}
%definition style pour mettre un fond blanc dans un node sans avoir des marges énormes
\tikzset{fondblanc/.style={ inner sep=2pt,fill=white,outer sep = 5pt}} 
\tikzset{fondblanc2/.style={ inner sep=2pt,fill=white}} 
\tikzset{fondblanc3/.style={ inner sep=1pt,fill=white,outer sep = 2pt}} 
\usetikzlibrary{calc,circuits.ee.IEC}
\usetikzlibrary{shapes}
\usepackage[european resistor, european voltage, european current]{circuitikz}
\usetikzlibrary{babel}
\usepackage{standalone}
\standaloneconfig{mode=buildnew}

\usepackage{picins}

% -------------------------------------
% Déclaration des titres
% -------------------------------------

\def\discipline{Sciences \\Industrielles de \\ l'Ingénieur}
\def\xxtete{Sciences Industrielles de l'Ingénieur}

\def\classe{Cy01 - R 02}
\def\xxnumpartie{Cycle 01}
\def\xxpartie{Modéliser les systèmes asservis -- Transformée de Laplace}

\def\xxnumchapitre{Révisions 2 \vspace{.2cm}}
\def\xxchapitre{\hspace{.12cm} Modéliser les systèmes asservis -- Transformée de Laplace}

\def\xxposongletx{2}
\def\xxposonglettext{1.45}
\def\xxposonglety{19}%16

\def\xxonglet{Cycle 01 -- Rév 2}

\def\xxactivite{Fiche}
\def\xxauteur{\textsl{Xavier Pessoles}}

\def\xxcompetences{%
\textsl{%
\textbf{Savoirs et compétences :}\\
}}

\def\xxfigures{
%incgraphics[width=.8\textwidth]{}%images/prot_01}
}%figues de la page de garde

\def\xxpied{%
Cycle 01 -- Modélisation des systèmes linéaires et non linéaires\\
Révision 2 -- \xxactivite%
}

\setcounter{secnumdepth}{5}
%---------------------------------------------------------------------------


\begin{document}
%\chapterimage{png/Fond_Cin}
\input{style/new_pagegarde}
\vspace{2cm}
\pagestyle{fancy}
\thispagestyle{plain}

\section{Définitions}

%\begin{itemize}[label=\ding{112},font=\color{ocre}] 
%\item Une
%\item Une
%\end{itemize}

\begin{defi}[Conditions de Heavisde -- Fonction causale -- Conditions initiales nulles] ~\\

Une fonction temporelle $f(t)$ vérifie les conditions de Heaviside lorsque les dérivées successives nécessaires à la résolution de l'équation différentielle sont nulles pour $t={0^{+}}$ :
$$
f({0^{+}})=0 \quad \dfrac{\d f({0^{+}})}{\d t} = 0 \quad \dfrac{\d^2f({0^{+}})}{\d t^2} = 0 ...
$$

On parle de conditions initiales nulles.

\end{defi}

\begin{defi}[Transformée de Laplace] ~\\
À toute fonction du temps $f(t)$, nulle pour $t\leq0$ (fonction causale), on fait correspondre une fonction $F(p)$ de la variable complexe $p$ telle que :
$$
\mathcal{L}\left[f(t)\right] = F(p)=\int\limits_{0^{+}}^\infty f(t)e^{-pt}\text{d}t.
$$
On note $\mathcal{L}\left[f(t)\right]$ la transformée directe et $\mathcal{L}^{-1}\left[F(p)\right]$ la transformée inverse.

\noindent De manière générale on note 
$\mathcal{L}\left[f(t)\right] = F(p)$,
$\mathcal{L}\left[e(t)\right] = E(p)$,
$\mathcal{L}\left[s(t)\right] = S(p)$,
$\mathcal{L}\left[\omega(t)\right] = \Omega(p)$,
$\mathcal{L}\left[\theta(t)\right] = \Theta(p)$ ...

\end{defi}


\begin{resultat}[Dérivation] ~\\

%La transformée de Laplace d'une dérivée seconde est donnée par : 
%$\mathcal{L}\left[ \dfrac{\d^2f(t)}{\d t^2}\right] =p^2F(p)-pf(0^+)-\dfrac{\d f(0^+)}{\d t}$

\noindent Dans les conditions de Heaviside :
$\mathbf{\mathcal{L}\left[ \dfrac{\text{\textbf{d}} f(t)}{\text{\textbf{d}} t}\right] =pF(p) 
 \quad
\mathcal{L}\left[ \dfrac{\text{\textbf{d}}^2f(t)}{\text{\textbf{d}} t^2}\right] =p^2F(p) 
 \quad
\mathcal{L}\left[ \dfrac{\text{\textbf{d}}^nf(t)}{\text{\textbf{d}} t^n}\right] =p^nF(p) }.$

\noindent\textit{\footnotesize{En dehors des conditions de Heaviside, la transformée de Laplace d'une dérivée première est donnée par $\mathcal{L}\left[ \dfrac{\d f(t)}{\d t}\right] =pF(p)-f(0^+)$.}}


\end{resultat}


\begin{defi}[Transformées usuelles] ~\\
\footnotesize{

\begin{center}
\begin{tabular}{|c|c||c|c|}
\hline
Domaine temporel $f(t)$ & Domaine de Laplace $F(p)$ & 
Domaine temporel $f(t)$ & Domaine de Laplace $F(p)$ \\
\hline
\hline
Dirac $\delta(t)$ &
$\mathbf{F(p)=1}$ &
Échelon $ u(t)=k $&
$ \mathbf{U(p) = \dfrac{k}{p}}$
\\
\hline
%Créneau $\forall t\in ]0,t_1 [ \quad f(t)= A$ & 
Fonction carrée $f(t)=t^2$& 
$\mathbf{F(p) =\dfrac{1}{p^2} }$ &
Puissances
$f(t) = t^n\cdot u(t)$ &
$F(p)=\dfrac{n!}{p^{n+1}} $
\\
\hline
$f(t) = \sin \left( \omega_0 t\right) \cdot u(t)$ &
$F(p) = \dfrac{\omega_0}{p^2+\omega_0^2} $ &
$f(t) = \cos \left( \omega_0 t\right) \cdot u(t)$ & 
$F(p) = \dfrac{p}{p^2+\omega_0^2} $ \\
\hline
$f(t)= e^{-at}\cdot u(t)$ & 
$F(p)= \dfrac{1}{p+a}$ &
$f(t) = e^{-at}\sin\left( \omega_0 t\right) \cdot u(t)$ &
$F(p)=\dfrac{\omega_0}{\left( p+a\right)^2 + \omega_0^2}$  \\
\hline
$f(t)$ est $T$ périodique &
$F(p)= \dfrac{\mathcal{L} \left[f_0 (t)\right]}{1-e^{-Tp}} \cdot u(t)$ &
$f(t)=t^ne^{-at}u(t)$ & $F(p)=\dfrac{n!}{\left( p+a\right)^{n+1}}$
\\
\hline
\end{tabular}
\end{center}}
\end{defi}


\section{Théorèmes}
\begin{multicols}{2}
\begin{theorem}[Théorème de la valeur initiale] ~\\
$$ \lim\limits_{t \to 0^+} f(t) = \lim\limits_{p \to \infty} pF(p)$$
\end{theorem} 

\begin{theorem}[Théorème de la valeur finale] ~\\
$$\lim\limits_{t \to \infty} f(t) = \lim\limits_{p \to 0} pF(p)$$
\end{theorem} 

\begin{theorem}[Théorème du retard] ~\\
$$\mathcal{L}\left[ f\left(t-t_0\right) \right] = e^{-t_0 p}F(p)$$
\end{theorem} 

\begin{theorem}[Théorème de l'amortissement] ~\\
$$\mathcal{L} \left[ e^{-a t} f\left(t\right) \right] = F(p+a)$$
\end{theorem} 
\end{multicols}


%\begin{center}
%\includestandalone{images/perf}
%\hfill
%\includestandalone{images/rampe}
%\end{center}



\end{document}


