\documentclass[10pt,fleqn]{article} % Default font size and left-justified equations
\usepackage[%
    pdftitle={Modélisation systèmes multiphysiques : Modélisation par fonction de transfert et schéma-blocs},
    pdfauthor={Xavier Pessoles}]{hyperref}

\input{style/new_style}
\input{style/macros_SII}

\fichetrue
%\fichefalse

%\proftrue
\proffalse

\tdtrue
%\tdfalse

%\courstrue
\coursfalse

%%%%%%%%%%%%%%%%%%%
\usepackage{numprint}
\usetikzlibrary{calc}
%definition style pour mettre un fond blanc dans un node sans avoir des marges énormes
\tikzset{fondblanc/.style={ inner sep=2pt,fill=white,outer sep = 5pt}} 
\tikzset{fondblanc2/.style={ inner sep=2pt,fill=white}} 
\tikzset{fondblanc3/.style={ inner sep=1pt,fill=white,outer sep = 2pt}} 
\usetikzlibrary{calc,circuits.ee.IEC}
\usetikzlibrary{shapes}
\usepackage[european resistor, european voltage, european current]{circuitikz}
\usetikzlibrary{babel}
\usepackage{standalone}
\standaloneconfig{mode=buildnew}
\usepackage{style/schemabloc}
\usepackage{picins}

% -------------------------------------
% Déclaration des titres
% -------------------------------------

\def\discipline{Sciences \\Industrielles de \\ l'Ingénieur}
\def\xxtete{Sciences Industrielles de l'Ingénieur}

\def\classe{Cy01 - R 03}
\def\xxnumpartie{Cycle 01}
\def\xxpartie{Modéliser le comportement linéaire et non linéaire des systèmes multiphysiques}

\def\xxnumchapitre{Révisions 3 \vspace{.2cm}}
\def\xxchapitre{\hspace{.12cm} Modélisation par fonction de transfert et schéma-blocs}

\def\xxposongletx{2}
\def\xxposonglettext{1.45}
\def\xxposonglety{19}%16

\def\xxonglet{Cycle 01 -- Rév 3}

\def\xxactivite{Application}
\def\xxtitreexo{Application}
\def\xxsourceexo{}
\def\xxauteur{\textsl{Xavier Pessoles}}

\def\xxcompetences{%
\textsl{%
\textbf{Savoirs et compétences :}\\
}}

\def\xxfigures{
%incgraphics[width=.8\textwidth]{}%images/prot_01}
}%figues de la page de garde

\def\xxpied{%
Cycle 01 -- Modéliser le comportement des systèmes multiphysiques\\
Révision 3 -- \xxactivite%
}

\setcounter{secnumdepth}{5}
%---------------------------------------------------------------------------


\begin{document}
%\chapterimage{png/Fond_Cin}
\input{style/new_pagegarde}
\vspace{6cm}
\pagestyle{fancy}
\thispagestyle{plain}

\section{Modélisation par schéma-blocs}
\begin{methode}
Dans le cas où vous ne savez pas comment démarrer, vous pouvez suivre la méthode suivante.
\begin{enumerate}
\item Identifier la grandeur physique d'entrée et la grandeur physique de sortie.
\item Lorsqu'une équation lie deux grandeurs physiques, réaliser le schéma-blocs associé à l'équation. 
\item Lorsqu'une équation lie trois grandeurs physiques, réaliser le schéma-blocs associé à l'équation en utilisant un comparateur.
\item Relier les blocs en commençant par l'entrée. Inverser les blocs si nécessaire.
\end{enumerate}
\end{methode}
\subsection{Modélisation du moteur à courant continu}
\subsection{Modélisation d'un système}
\section{Réduction de schéma-blocs}
\textit{D'après ressources de V. Reydellet.}

\begin{center}
\includegraphics[width=\linewidth]{images/sb_01}
\end{center}

\begin{center}
\includegraphics[width=\linewidth]{images/sb_02}
\end{center}

\begin{center}
\includegraphics[width=\linewidth]{images/sb_03}
\end{center}

\begin{center}
\includegraphics[width=\linewidth]{images/sb_04}
\end{center}

\end{document}


