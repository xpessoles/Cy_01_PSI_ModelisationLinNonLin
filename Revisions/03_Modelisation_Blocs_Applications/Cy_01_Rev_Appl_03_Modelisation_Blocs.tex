\documentclass[10pt,fleqn]{article} % Default font size and left-justified equations
\usepackage[%
    pdftitle={Modélisation systèmes multiphysiques : Modélisation par fonction de transfert et schéma-blocs},
    pdfauthor={Xavier Pessoles}]{hyperref}

\input{style/new_style}
\input{style/macros_SII}

\fichetrue
%\fichefalse

%\proftrue
\proffalse

\tdtrue
%\tdfalse

%\courstrue
\coursfalse

%%%%%%%%%%%%%%%%%%%
\usepackage{numprint}
\usetikzlibrary{calc}
%definition style pour mettre un fond blanc dans un node sans avoir des marges énormes
\tikzset{fondblanc/.style={ inner sep=2pt,fill=white,outer sep = 5pt}} 
\tikzset{fondblanc2/.style={ inner sep=2pt,fill=white}} 
\tikzset{fondblanc3/.style={ inner sep=1pt,fill=white,outer sep = 2pt}} 
\usetikzlibrary{calc,circuits.ee.IEC}
\usetikzlibrary{shapes}
\usepackage[european resistor, european voltage, european current]{circuitikz}
\usetikzlibrary{babel}
\usepackage{standalone}
\standaloneconfig{mode=buildnew}
\usepackage{style/schemabloc}
\usepackage{picins}

% -------------------------------------
% Déclaration des titres
% -------------------------------------

\def\discipline{Sciences \\Industrielles de \\ l'Ingénieur}
\def\xxtete{Sciences Industrielles de l'Ingénieur}

\def\classe{Cy01 - R 03}
\def\xxnumpartie{Cycle 01}
\def\xxpartie{Modéliser le comportement linéaire et non linéaire des systèmes multiphysiques}

\def\xxnumchapitre{Révisions 3 \vspace{.2cm}}
\def\xxchapitre{\hspace{.12cm} Modélisation par fonction de transfert et schéma-blocs}

\def\xxposongletx{2}
\def\xxposonglettext{1.45}
\def\xxposonglety{10}%19

\def\xxonglet{Cycle 01 -- Rév 3}

\def\xxactivite{Application}
\def\xxtitreexo{Application}
\def\xxsourceexo{}
\def\xxauteur{\textsl{Xavier Pessoles}}

\def\xxcompetences{%
\textsl{%
\textbf{Savoirs et compétences :}\\
}}

\def\xxfigures{
%incgraphics[width=.8\textwidth]{}%images/prot_01}
}%figues de la page de garde

\def\xxpied{%
Cycle 01 -- Modéliser le comportement des systèmes multiphysiques\\
Révision 3 -- \xxactivite%
}

\setcounter{secnumdepth}{5}
%---------------------------------------------------------------------------


\begin{document}
%\chapterimage{images/Fond_Cin}
\input{style/new_pagegarde}
\vspace{6cm}
\pagestyle{fancy}
\thispagestyle{plain}

\def\columnseprulecolor{\color{ocre}}
\setlength{\columnseprule}{0.4pt} 


\begin{multicols}{2}
\section*{Modélisation par schéma-blocs}

\begin{methode}
Dans le cas où vous ne savez pas comment démarrer, vous pouvez suivre la méthode suivante.
\begin{enumerate}
\item Identifier la grandeur physique d'entrée et la grandeur physique de sortie.
\item Lorsqu'une équation lie deux grandeurs physiques, réaliser le schéma-blocs associé à l'équation. 
\item Lorsqu'une équation lie trois grandeurs physiques, réaliser le schéma-blocs associé à l'équation en utilisant un comparateur.
\item Relier les blocs en commençant par l'entrée. Inverser les blocs si nécessaire.
\end{enumerate}
\end{methode}
\subsection*{Modélisation du moteur à courant continu}
On donne les équations du moteur à courant continu :
\begin{itemize}
\item $u(t) = e(t)+ Ri(t) L \dfrac{\text{d}i(t)}{\text{d} t}$;
\item $e(t)=K\omega(t)$;
\item $c(t)=Ki(t)$;
\item $c(t)-c_r(t) - f\omega(t)=J\dfrac{\text{d}\omega(t)}{\text{d} t}$.
\end{itemize}
\subparagraph*{}
\textit{Réaliser le schéma-blocs. Exprimer $\Omega(p)$ en fonction de $U(p)$ et $C_r(p)$.}

\subsection*{Modélisation d'une servo commande}

\begin{center}
\includegraphics[width=.9\linewidth]{images/img5}
\end{center}

Les différentes équations temporelles qui modélisent le fonctionnement d'une servocommande sont :
\begin{itemize}
\item un amplificateur différentiel défini par : $u_c(t)=\dfrac{i(t)}{K_a}+u_s(t)$;
\item débit dans le vérin dans le cas d'une hypothèse de fluide incompressible $q(t)=S\cdot\dfrac{dx(t)}{dt}$;
\item capteur de position : $u_s(t)=K_c\cdot x(t)$;
\item le servo-distributeur est un composant de la chaîne de commande conçu pour fournir un débit hydraulique $q(t)$ proportionnel au courant de commande $i(t)$. (Attention, valable uniquement en régime permanent.) Le constructeur fournit sa fonction de transfert :
$$
F(p)=\dfrac{Q(p)}{I(p)}=\dfrac{K_d}{1+Tp}
$$
où $K_d$ est le gain du servo-distributeur et $T$ sa constante de temps.
\end{itemize}

\subparagraph*{}
\textit{Réaliser le schéma-blocs.}
\ifprof

\begin{corrige}
On a :
\begin{itemize}
\item $U_c(p)=\dfrac{1}{K_a}I(p)+U_s(p)$
\item $Q(p)=SpX(p)$
\item $U_S(p)=K_C\cdot X(p)$
\item $F(p)=\dfrac{Q(p)}{I(p)}=\dfrac{K_d}{1+Tp}$
\end{itemize}


\begin{minipage}[c]{.23\linewidth}
\begin{center}
\includegraphics[width=.95\textwidth]{images/bloc1}
\end{center}
\end{minipage}\hfill
\begin{minipage}[c]{.23\linewidth}
\begin{center}
\includegraphics[width=.95\textwidth]{images/bloc2}
\end{center}
\end{minipage}\hfill
\begin{minipage}[c]{.23\linewidth}
\begin{center}
\includegraphics[width=.95\textwidth]{images/bloc3}
\end{center}
\end{minipage}\hfill
\begin{minipage}[c]{.23\linewidth}
\begin{center}
\includegraphics[width=.95\textwidth]{images/bloc4}
\end{center}
\end{minipage}



\begin{center}
\includegraphics[width=.8\textwidth]{images/schema_bloc}
\end{center}
\end{corrige}
\else 
\fi




\section*{Réduction de schéma-blocs}
\textit{D'après ressources de V. Reydellet.}
\subparagraph*{}
\textit{Réduire les schéma-blocs suivants.}
\begin{center}
\includegraphics[scale=.2]{images/sb_01}
\end{center}

\begin{center}
\includegraphics[scale=.2]{images/sb_02}
\end{center}

\section*{Transformation de schéma-blocs}
\subparagraph*{}
\textit{Transformer le schéma-bloc suivant pour obtenir un schéma-blocs à retour unitaire.}
\begin{center}
\includegraphics[scale=.2]{images/sb_03}
\end{center}

\subparagraph*{}
\textit{Modifier le schéma-blocs suivant pour obtenir la forme proposée. Déterminer ensuite l'expression de $S(p)$ en fonction de $E(p)$ et $P(p)$.}
\begin{center}
\includegraphics[scale=.2]{images/sb_04}
\end{center}

\begin{center}
\includegraphics[scale=.2]{images/sb_05}
\end{center}

\end{multicols}
\end{document}


