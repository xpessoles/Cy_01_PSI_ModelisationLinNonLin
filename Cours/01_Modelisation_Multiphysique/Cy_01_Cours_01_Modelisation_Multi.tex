\documentclass[10pt,fleqn]{article} % Default font size and left-justified equations
\usepackage[%
    pdftitle={Modélisation systèmes multiphysiques : Modélisation linéaire et non linéaire},
    pdfauthor={Xavier Pessoles}]{hyperref}

\input{style/new_style}
\input{style/macros_SII}

\fichetrue
\fichefalse

\proftrue
%\proffalse

%\tdtrue
\tdfalse

\courstrue
%\coursfalse

%%%%%%%%%%%%%%%%%%%
\usepackage{numprint}
\usetikzlibrary{calc}
%definition style pour mettre un fond blanc dans un node sans avoir des marges énormes
\tikzset{fondblanc/.style={ inner sep=2pt,fill=white,outer sep = 5pt}} 
\tikzset{fondblanc2/.style={ inner sep=2pt,fill=white}} 
\tikzset{fondblanc3/.style={ inner sep=1pt,fill=white,outer sep = 2pt}} 
\usetikzlibrary{calc,circuits.ee.IEC}
\usetikzlibrary{shapes}
\usepackage[european resistor, european voltage, european current]{circuitikz}
\usetikzlibrary{babel}
\usepackage{standalone}
\standaloneconfig{mode=buildnew}
\usepackage{siunitx}
\usepackage{picins}

% -------------------------------------
% Déclaration des titres
% -------------------------------------

\def\discipline{Sciences \\Industrielles de \\ l'Ingénieur}
\def\xxtete{Sciences Industrielles de l'Ingénieur}

\def\classe{\textsf{Cy 01}}
\def\xxnumpartie{Cycle 01}
\def\xxpartie{Modéliser le comportement linéaire et non linéaire des systèmes multiphysiques}

\def\xxnumchapitre{Chapitre 1 \vspace{.2cm}}
\def\xxchapitre{\hspace{.12cm} Modélisation multiphysique}

\def\xxposongletx{2}
\def\xxposonglettext{1.45}
\def\xxposonglety{19}%16

\def\xxonglet{Cycle 01}

\def\xxactivite{Cours}
\def\xxauteur{\textsl{Xavier Pessoles}}

\def\xxcompetences{%
\textsl{%
\textbf{Savoirs et compétences :}\\
}}

\def\xxfigures{
\includegraphics[width=.8\textwidth]{png/Header_Peugeot}%images/prot_01}
}%figues de la page de garde

\def\xxpied{%
Cycle 01 -- Modéliser le comportement des systèmes multiphysiques\\
Chapitre 1 -- \xxactivite%
}

\setcounter{secnumdepth}{5}
%---------------------------------------------------------------------------


\begin{document}
\chapterimage{png/Header_Peugeot}
\input{style/new_pagegarde}
\setlength{\columnseprule}{.1pt}

\vspace{2cm}
\pagestyle{fancy}
\thispagestyle{plain}
\section{Introduction}
\subsection{Qu'est-ce qu'un système multiphysique}

Pour comprendre le fonctionnement des systèmes qui nous entourent, il est souvent nécessaire de maîtriser un voire plusieurs domaines de la physique. En effet, le winch utilisé dans le laboratoire a un fonctionnement essentiellement mécanique. En revanche, le simulateur de drone $D^2C$ est composé d'une partie mécanique (rotation du banc et des hélices) une partie électrotechnique (moteurs) une partie électronique (commande des moteurs) une partie informatique (gestion de la commande et des informations). 

Pour modéliser un système, plusieurs outils peuvent être nécessaires. Lorsqu'un outil est associé à un champ de la physique, on peut parler de modèle << mono physique >> :
\begin{itemize}
\item pour modéliser la géométrie d'un système ou le comportement d'un mécanisme, on peut faire appel à SolidWorks par exemple;
\item pour modéliser la partie électrique d'un système il est possible d'utiliser un logiciel comme PSpice;
\item pour programmer une interface graphique d'un logiciel, il est possible d'utiliser Python...
\end{itemize}

En revanche, lorsqu'on veut que tous ces domaines communiquent, il faut une plateforme commune permettant l'échange entre les modèles. On parle alors de modélisation multiphysique. Il est possible d'utiliser des logiciels comme Scilab (Xcos -- Modelica) ou Matlab (Simulink -- Simscape). 



\subsection{Pourquoi modéliser des systèmes ?}
Dans l'industrie, les modèles sont indispensables. Ils permettent d'avoir un modèle numérique, image du produit que l'on cherche à réaliser ou que l'on a déjà. L'image doit être aussi fidèle à la réalité que possible. On a vu que ce modèle peut-être << monophysique >> ou << multiphysique >>. 

L'objectif du modèle est de se substituer au produit réel. Les simulations réalisées sur le modèle ont pour objectif de remplacer des expérimentations sur le produits, considérées comme coûteuse en temps et en argent. 

Il est possible de recenser les avantages et inconvénients liés à la simulation des modèles \cite{1}. 
\begin{multicols}{2}
\begin{itemize}[label=\ding{51}]
\item Pouvoir prévoir le comportement du système réel alors qu'il n’existe pas encore lors de la phase de conception;
\item permettre la prévision de phénomènes (en météorologie par exemple);
\item éviter ou limiter le recours aux expérimentations réelles qui peuvent être très
coûteuses ou très dangereuses, voire proscrites (essais nucléaires militaires) ou
impossibles dans l’état actuel des connaissances et des moyens (projet ITER) ;
\item quand l’échelle de temps des phénomènes dans le système réel ne permet pas une
expérience « en une durée raisonnable » pour effectuer des observations ou des mesures.
(premiers instants de l’univers ($t < 10^{-6} \text{s}$) ou l'évolution des galaxies
($t>10^6$ années);
\item « observer » ou représenter des variables inaccessibles à l'expérience ou la mesure;
\item les manipulations sont faciles sur un modèle. Elles peuvent être répétées, voire itérées
automatiquement pour apprécier de très nombreuses situations ;
\item le droit à l’erreur, sans risque ;
\item la possibilité de supprimer des phénomènes perturbateurs ou des effets
secondaires.
\end{itemize}
\vfill\null
\columnbreak

\begin{itemize}[label=\ding{56}]
\item Avoir une confiance aveugle dans les simulations et ses résultats : des erreurs liées aux
modèles ou aux calculs peuvent ne pas être perçues immédiatement ;
\item « oublier » les conditions de la simulation et les hypothèses formulées pour établir le
modèle et surtout dans le cas des systèmes complexes ;
\item « inverser » la réalité et « forcer » le réel à intégrer les contraintes du modèle ;
\item oublier le niveau de précision des résultats provenant du modèle.
\end{itemize}
\end{multicols}


\newpage

\section{Modélisation des systèmes multiphysiques}
\subsection{Modélisation causale et acausale}

\begin{multicols}{2}
Lorsque le fonctionnement d'un système est régit par une équation différentielle, dont l'ordre de dérivation de la sortie est supérieur à l'ordre de dérivation de l'entrée, la sortie est une conséquence de l'entrée. En passant l'équation dans le domaine de Laplace puis en la traduisant sous forme de schéma bloc, on obtient alors un bloc \textbf{orienté} traduisant ainsi la relation de cause à effet entre l'entrée et la sortie. 

On parle ici de modélisation \textbf{causale}. 

Les liens entre les blocs représentent une grandeur physique (courant, tension, position, vitesse \textit{etc}.).

\vfill\null
\columnbreak

En modélisation acausale, les entrées et sorties ne sont pas spécifiées. Les liens entre entrées et sorties sont définies de manière implicite. Lorsqu'on visualise la traduction graphique d'un modèle acausal, les liens ne sont pas orientés (les blocs sont << réversibles >>). Les blocs sont traversés par des flux d'énergie d'un même domaine physique.  

Dans Matlab -- Simulink, on parle de grandeurs potentielles (across) et de grandeur traversante (through) : 
\begin{itemize}
\item une variable potentielle est mesurée par un instrument
en parallèle avec la chaîne d’énergie;
\item une variable traversante est mesurée par un instrument
en série avec la chaîne d’énergie.
\end{itemize}

Dans Scilab -- Coselica, (langage Modelica), on parle de variables potentielles et flux.
\begin{itemize}
\item variables potentielles: les variables qui sont reliées au même port sont égales;
\item variables flux : les variables qui aboutissent au
même port ont pour somme 0.
\end{itemize}
\end{multicols}


\begin{center}
\begin{tabular}{ccc}
\hline
Domaine physique & Variables << across >> &  Variables << through >> \\ \hline\hline
Électrique & Tension (\si{V})& Courant (\si{A})\\ \hline
Hydraulique & Pression (\si{Pa})& Débit (\si{m^3.s^{-1}})\\ \hline
Mécanique de translation & Vitesse linéaire (\si{m.s^{-1}})& Force (\si{N})\\ \hline 
Mécanique de rotation & Vitesse angulaire(\si{rad.s^{-1}}) & Moment (\si{Nm})\\ \hline
%Pneumatique & Pression (V)& Débit massique (V)et flux d'entropie \\ \hline 
Thermique & Température (V)& Flux thermique et flux d'entropie \\ \hline
\\
\end{tabular}

\textit{Modélisation acausale dans Matlab -- Simulink -- Simscape : variables << across >> et << through >>.}
\end{center}

\subsection{Les différents modèles et outils}

La figure ci-dessous présente un modèle causal et un modèle acausal du système de laboratoire << Control'X >>en utilisant le logiciel Matlab-Simulink.

\begin{center}
\includegraphics[width=\linewidth]{images/Modeles}
\end{center}

Visuellement on constate que sur le modèle causal, les composants du système apparaissent. Ainsi, sans connaitre les lois de comportements des composants, il est possible de réaliser le modèle multiphysique d'un système.

En modélisation acausale, on utilise une représentation par schéma-blocs. Il est ici indispensable de connaitre les modèles de connaissance ou de comportement des composants pour réaliser le modèle.

\subsection{Résolution -- Avantage et Inconvénients}

\noindent
\begin{minipage}[c]{.6\linewidth}
Que ce soient des modèles causaux ou acausaux, Matlab a recours à des solveurs pour simuler le comportement des systèmes. En effet, des équations différentielles, linéaires ou non, sont << cachées >> derrière les blocs.  

Par défaut, nous laisserons un choix automatique du solveur. Cependant, certains modèles imposeront (via un message d'erreur) le changement de ce solveur. Par ailleurs le pas de simulation devra être changé dans certain cas, dans le but d'améliorer la précision des résultats.
\end{minipage} \hfill
\begin{minipage}[c]{.35\linewidth}
\begin{center}
\includegraphics[width=\linewidth]{images/solveur}
\end{center}
\end{minipage}

\begin{warn}
Attention, il est à noter qu'il n'est possible de réaliser de diagramme de Bode en utilisant un modèle acausal. Ceci peut être un handicap en phase de conception d'un système car il devient plus délicat de déterminer les résonances du système.
\end{warn}
\section{Modélisation des systèmes physiques}
On utilisera ici Matlab -- Simulink. Certains modèles sont aussi transposables avec Scilab -- Xcos. 

\subsection{Modélisation des systèmes mécaniques}
Deux modes de modélisation des systèmes mécanique est disponible : une modélisation 1D (mouvement de translation suivant une seule direction, mouvement de rotation autour d'un axe fixe) ou une modélisation 3D. 
La modélisation 1D est possible directement dans Matlab. Pour la modélisation 3D, il est plus aisé d'utiliser SolidWorks. 

\begin{center}
\includegraphics[width=.6\linewidth]{images/Masse_Ressort}

\textit{Modélisation causale et acausale d'un système avec deux systèmes << masse -- ressort -- amortisseur >> en série.}
\end{center}

\begin{center}
\includegraphics[width=.9\linewidth]{images/Modele3D}

\textit{Modélisation acausale 3D d'un bras robotisé.}
\end{center}

\subsection{Modélisation des systèmes électriques}

\begin{center}
\includegraphics[width=.6\linewidth]{images/MoteurCC}

\textit{Modélisation acausale de la commande d'un moteur à courant continu.}
\end{center}


\subsection{Modélisation des systèmes thermiques, pneumatiques et hydrauliques}
En cas de besoin, des exemples complémentaires sont disponibles en utilisant Maltab -- Simulink.

%\subsection{Modélisation des interfaces}

\section{Modélisation des non-linéarités}

\subsection{Linéarisation}
\subsection{Seuil}
\subsection{Saturation}
\subsection{Hystérésis}

\section{Modélisation des systèmes numériques}
Effet du pas de temps.

A voir ultérieurement. 


\begin{thebibliography}{2}
   \bibitem[1]{ref1} Y. Crevits, {\it Éléments de modélisation multi-physique des systèmes industriels en vue de leur simulation numérique, Juin 2015.}
   \bibitem[2]{ref2} P. Fichou, {\it La modélisation multiphysique, Technologie, Mars 2012.}
   \bibitem[3]{ref3} Yvan Liebgott, {\it Modélisation et Simulation des Systemes Multi-Physiques avec MATLAB.}
\end{thebibliography}

\end{document}



